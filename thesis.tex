%
% Template Laporan Skripsi/Thesis Universitas Indonesia
%
% @author  Ichlasul Affan, Azhar Kurnia
% @version 2.1.3
%
% Dokumen ini dibuat berdasarkan standar IEEE dalam membuat class untuk
% LaTeX dan konfigurasi LaTeX yang digunakan Fahrurrozi Rahman ketika
% membuat laporan skripsi, yang kemudian diadaptasi oleh Andreas Febrian dan
% Lia Sadita untuk template skripsi tahun 2010.
% Konfigurasi template sebelumnya telah disesuaikan dengan
% aturan penulisan thesis yang dikeluarkan UI pada tahun 2017.
%

%
% Tipe dokumen adalah report dengan satu kolom.
%
\documentclass[12pt, a4paper, onecolumn, twoside, final]{report}
\raggedbottom

% Load konfigurasi LaTeX untuk tipe laporan thesis
\usepackage{_internals/uithesis}
%


% Load konfigurasi khusus untuk laporan yang sedang dibuat
%-----------------------------------------------------------------------------%
% Judul Dokumen
%-----------------------------------------------------------------------------%
%
% Judul laporan.
\def\judul{Judul Karya Ilmiah Anda}
%
% Tulis kembali judul laporan namun dengan bahasa Ingris
\def\judulInggris{Your Scientific Publication Title}


%-----------------------------------------------------------------------------%
% Tipe Dokumen
%-----------------------------------------------------------------------------%
%
% Tipe laporan, dapat berisi: Laporan Kerja Praktik, Kampus Merdeka, Skripsi, Tugas Akhir, Thesis, atau Disertasi
\def\type{Skripsi}
%
% Nama jalur Kampus Merdeka (hanya perlu diisi jika tipe laporan adalah Kampus Merdeka
% Contoh isian (khusus Fasilkom): Studi Independen, Magang Mitra, Magang BUMN, Bangkit, Apple Academy, BYOC
\def\kampusMerdekaType{}
% Jika ada perwakilan mitra, isi dengan jabatan perwakilan mitra tersebut
% (misal: Cohort Manager)
% Kosongkan jika tidak ada perwakilan mitra
\def\partnerPosition{}
% Jika ada perwakilan mitra, isi dengan nama perusahaan atau nama program
% (misal: PT. Astra International, Bangkit Academy 2023)
% Kosongkan jika tidak ada perwakilan mitra
\def\partnerInstance{}
%
% Jenjang studi, dapat berisi: Diploma, Sarjana, Magister, atau Doktor
\def\jenjang{Sarjana}


%-----------------------------------------------------------------------------%
% Informasi Penulis
%-----------------------------------------------------------------------------%
%
% Tulis nama Anda
% Kosongkan penulisDua dan penulisTiga jika Anda melaksanakan tugas akhir/laporan secara individu
\def\penulisSatu{Farel Musyaffa Arya Putra Majesta} % nama lengkap penulis pertama
\def\penulisDua{} % nama lengkap penulis kedua
\def\penulisTiga{} % nama lengkap penulis ketiga
%
% Tulis NPM Anda
% Kosongkan npmDua dan npmTiga jika Anda melaksanakan tugas akhir/laporan secara individu
\def\npmSatu{2006596472} % NPM penulis pertama
\def\npmDua{} % NPM penulis kedua
\def\npmTiga{} % NPM penulis ketiga
%
% Tulis Program Studi yang Anda ambil
% Kosongkan programDua dan programTiga jika Anda melaksanakan tugas akhir/laporan secara individu
\def\programSatu{Ilmu Komputer} % program studi penulis pertama
\def\programDua{} % program studi penulis kedua
\def\programTiga{} % program studi penulis ketiga
%
% Tulis Program Studi yang Anda ambil dalam bahasa inggris
% Kosongkan programDua dan programTiga jika Anda melaksanakan tugas akhir/laporan secara individu
\def\studyProgramSatu{Computer Science} % 1st author's study program
\def\studyProgramDua{} % 2nd author's study program
\def\studyProgramTiga{} % 3rd author's study program


%-----------------------------------------------------------------------------%
% Informasi Dosen Pembimbing & Penguji
%-----------------------------------------------------------------------------%
%
% Tuliskan pembimbing
% Untuk Kampus Merdeka: Tuliskan dosen PIC/pembimbing dari Fakultas Anda
\def\pembimbingSatu{Pembimbing Pertama Anda}
% S1 s.d. S3: Kosongkan jika tidak ada pembimbing kedua
% Untuk Kampus Merdeka: Tuliskan penanggung jawab/penyelia/mitra
%                       dari program Kampus Merdeka yang Anda ambil (jika ada)
\def\pembimbingDua{}
% S2 & S3: Kosongkan jika tidak ada pembimbing ketiga
\def\pembimbingTiga{}

%
% Tuliskan penguji
\def\pengujiSatu{Penguji Pertama Anda}
\def\pengujiDua{Penguji Kedua Anda}
% Kosongkan jika tidak ada penguji ketiga (umumnya penguji ketiga hanya ada untuk S2)
\def\pengujiTiga{}
% Kosongkan jika tidak ada penguji keempat, kelima, atau keenam (umumnya penguji > 3 hanya ada untuk S3)
\def\pengujiEmpat{}
\def\pengujiLima{}
\def\pengujiEnam{}


%-----------------------------------------------------------------------------%
% Informasi Lain (Asal Fakultas, Tanggal, dsb.)
%-----------------------------------------------------------------------------%
%
% Tuliskan Fakultas dimana penulis berada
\def\fakultas{Fakultas Ilmu Komputer}
%
% Tuliskan bulan dan tahun publikasi laporan
\Var{\bulanTahun}{Bulan Tahun}
%
% Tuliskan gelar yang akan diperoleh dengan menyerahkan laporan ini
\def\gelar{Gelar Jurusan Anda}
%
% Tuliskan tanggal pengesahan laporan, waktu dimana laporan diserahkan ke
% penguji/sekretariat
\def\tanggalSiapSidang{Tanggal Bulan Tahun}
%
% Tuliskan tanggal keputusan sidang dikeluarkan dan penulis dinyatakan
% lulus/tidak lulus
\def\tanggalLulus{Tanggal Bulan Tahun}


%-----------------------------------------------------------------------------%
% Judul Setiap Bab
%-----------------------------------------------------------------------------%
%
% Berikut ada judul-judul setiap bab.
% Silahkan diubah sesuai dengan kebutuhan.
%
\Var{\kataPengantar}{Kata Pengantar}
\Var{\babSatu}{Pendahuluan}
\Var{\babDua}{Kerangka Berpikir}
\Var{\babTiga}{Penggunaan Lanjutan}
\Var{\babEmpat}{Struktur Template}
\Var{\babLima}{Kasus-Kasus Khusus}
\Var{\kesimpulan}{Penutup}


%-----------------------------------------------------------------------------%
% Capitalized Variables
% Anda tidak perlu mengubah apapun di bagian ini
%-----------------------------------------------------------------------------%
\Var{\Judul}{\judul}
\Var{\Type}{\type}
\Var{\PenulisSatu}{\penulisSatu}
\Var{\PenulisDua}{\penulisDua}
\Var{\PenulisTiga}{\penulisTiga}
\Var{\Fakultas}{\fakultas}
\Var{\ProgramSatu}{\programSatu}
\Var{\ProgramDua}{\programDua}
\Var{\ProgramTiga}{\programTiga}



% Daftar pemenggalan suku kata dan istilah dalam LaTeX
\include{_internals/hypeindonesia}
% Daftar istilah yang mungkin perlu ditandai
\input{config/istilah}

% Awal bagian penulisan laporan
\begin{document}
%
% Sampul Laporan
\include{_internals/sampul}
% \forceclearchapter

%
% Gunakan penomeran romawi
\pagenumbering{roman}
%
% Menghilangkan penebalan pada huruf-huruf table of content
% dari halaman judul hingga daftar lampiran
\disableboldchapterintoc
%
% load halaman judul dalam
% \strcompare{Kampus Merdeka}{\type}{} {
% 	\addChapter{HALAMAN JUDUL}
% 	\include{_internals/judul_dalam}
% 	\forceclearchapter
% }

%
% load halaman orisinalitas

% Menghilangkan penomoran
\pagenumbering{gobble}

\strcompare{Laporan Kerja Praktik}{\type}{} {
\strcompare{Kampus Merdeka}{\type}{} {
	% \include{src/00-frontMatter/pernyataanOrisinalitas}
	% \forceclearchapter
}}

% Memunculkan penomoran kembali
\pagenumbering{roman}

%
% setelah bagian ini, halaman dihitung sebagai halaman ke 2
\setcounter{page}{2}

%
% Lembar Penegesahan
\strcompare{Laporan Kerja Praktik}{\type}
{
	% Lembar Pengesahan Kerja Praktik dari LaTeX
	% \addChapter{LEMBAR PERSETUJUAN DOSEN KERJA PRAKTIK}
	% \include{src/00-frontMatter/pengesahanKP}
	% \forceclearchapter

}{
\strcompare{Kampus Merdeka}{\type}
{
	% \addChapter{LEMBAR PENGESAHAN}
	% \include{src/00-frontMatter/pengesahanMBKM}
	% \forceclearchapter
}
{
	% \addChapter{LEMBAR PENGESAHAN}
	% Gunakan salah satu (comment atau hapus kode yang tidak perlu):
	% Lembar Pengesahan Tugas Akhir dari LaTeX
	% \strcompare{Doktor}{\jenjang}
	% {\include{src/00-frontMatter/pengesahanSidangS3}}
	% {\include{src/00-frontMatter/pengesahanSidang}}
	% \forceclearchapter
	% Lembar Pengesahan dari PDF lain (misal: generated oleh SISIDANG [Fasilkom])
	%\putpdf{assets/pdfs/pengesahanSidang.pdf}
}}


\strcompare{Laporan Kerja Praktik}{\type}{} {
\strcompare{Kampus Merdeka}{\type}{} {
	% %
	% % Kata Pengantar
	% \addChapter{\kataPengantar}
	% \include{src/00-frontMatter/kataPengantar}
	% \forceclearchapter
	% %
	% % Lembar Persetujuan Publikasi Ilmiah
	% \addChapter{LEMBAR PERSETUJUAN PUBLIKASI ILMIAH}
	% \include{src/00-frontMatter/persetujuanPublikasi}
	% \forceclearchapter
}}

%
% Untuk halaman pertama setiap chapter mulai dari abstrak, tetap berikan mark universitas.
%
\pagestyle{first-pages}

%
% \addChapter{ABSTRAK}
% \include{src/00-frontMatter/abstrak}
% %
% %
% \include{src/00-frontMatter/abstract}

%
% Daftar isi, gambar, tabel, dan kode
%
% \CAPinToC % All entries in ToC will be CAPITALIZED from here on
% \phantomsection %hack to make them clickable
% \singlespacing
% \tableofcontents
% \setstretch{1.4}
% \clearpage
% \phantomsection %hack to make them clickable
% \singlespacing
% \listoffigures
% \setstretch{1.4}
% \clearpage
% \phantomsection %hack to make them clickable
% \singlespacing
% \listoftables
% \setstretch{1.4}
% \clearpage

%
% Daftar Kode Program
% Comment to disable.
%
% \phantomsection %hack to make them clickable
% \addcontentsline{toc}{chapter}{\lstlistlistingname}
% \singlespacing
% \listoflistings
% \setstretch{1.4}
% \clearpage

%
% Daftar Isi yang Didefinisikan Sendiri (Custom)
% Definisi jenis objek baru dapat dilakukan di uithesis.sty
% Uncomment to use.
%
%\phantomsection %hack to make them clickable
%\addcontentsline{toc}{chapter}{\listofthingname}
%\singlespacing
%\listofthing
%\setstretch{1.4}
%\clearpage

%
% Daftar Equation (Persamaan Matematis)
% Uncomment to use.
%
% \phantomsection %hack to make them clickable
% \addcontentsline{toc}{chapter}{\listofequname}
% \singlespacing
% \listofequ
% \setstretch{1.4}
% \clearpage

%
% Daftar Lampiran
% Comment to disable.
%
% \phantomsection %hack to make them clickable
% \addcontentsline{toc}{chapter}{\listofappendixname}
% \singlespacing
% \listofappendix
% \setstretch{1.4}

% % Table of content normal lagi hurufnya
% \enableboldchapterintoc

% \clearpage

% Jika penomoran romawi selesai di ganjil
%\naiveoddclearchapter
% Jika penomoran romawi selesai di genap
%\naiveevenclearchapter

\noCAPinToC % Revert to original \addcontentsline formatting

%
% Gunakan penomeran Arab (1, 2, 3, ...) setelah bagian ini.
%
\pagenumbering{arabic}
\pagestyle{standard}
% \setlength{\belowcaptionskip}{+2pt}


\setoddevenheader
%-----------------------------------------------------------------------------%
\chapter{\babSatu}
\label{bab:1}
%-----------------------------------------------------------------------------%


%-----------------------------------------------------------------------------%
\section{Latar Belakang}
\label{sec:latarBelakang}
%-----------------------------------------------------------------------------%
Peran dari internet di abad ke-21 sudah tidak bisa dipisahkan lagi dari kehidupan sehari-hari manusia. Perkembangan teknologi yang terjadi
dalam bidang teknologi informasi memungkinkan terjadinya peningkatan pesat terhadap aksesibilitas dari perangkat-perangkat yang memungkinkan kita untuk
dapat terhubung satu sama lain meskipun berada dalam jarak fisik yang jauh melalui jaringan internet. Menurut survei yang dilakukan oleh APJII (2024),
terdapat sebanyak 221 juta pengguna internet di Indonesia, yang berarti penetrasi pengguna internet di Indonesia telah mencapai angka 79,5\% dari total
jumlah penduduk Indonesia. 
\par

Relevansi dari internet dalam kehidupan sehari-hari penduduk Indonesia menyebabkan jaringan komputer menjadi salah satu aspek yang penting dalam perkembangan
teknologi informasi di Indonesia kedepannya. Untuk itu, Indonesia perlu menyiapkan sumber daya manusia yang memiliki kompetensi yang memadai dalam bidang
jaringan komputer. Fakultas Ilmu Komputer Universitas Indonesia telah menyediakan sarana untuk pemberdayaan kompetensi tersebut melalui pengadaan
mata kuliah Jaringan Komputer dan Jaringan Komputer Lanjut yang ditawarkan kepada mahasiswa Universitas Indonesia. Mata kuliah Jaringan Komputer Lanjut
hadir sebagai lanjutan dari mata kuliah Jaringan Komputer, dimana mahasiswa Jaringan Komputer Lanjut diharapkan telah memahami dan memiliki kompetensi
yang memadai dalam materi-materi yang telah disediakan dalam mata kuliah Jaringan Komputer.
\par

Memastikan bahwa mahasiswa Jaringan Komputer Lanjut memiliki kompetensi yang memadai dalam materi mata kuliah Jaringan Komputer dapat dilakukan dengan berbagai macam cara.
Dalam waktu penulisan, mata kuliah Jaringan Komputer Lanjut biasanya mengalokasikan waktu satu minggu sebagai waktu untuk meninjau ulang materi mata kuliah
Jaringan Komputer, dimana mahasiswa akan diberikan materi singkat untuk mengingat kembali mata kuliah Jaringan Komputer serta akan diberikan sebuah evaluasi
dalam bentuk kuis untuk mengetes sebarapa jauh mahasiswa memahami materi Jaringan Komputer. Evaluasi kuis ini dilaksanakan dalam bentuk tes tertulis, yang nantinya
akan dinilai oleh dosen secara manual. 
\par

Meskipun proses tersebut merupakan sebuah solusi yang memadai untuk meninjau kompetensi mahasiswa, tentunya proses tersebut juga membutuhkan waktu yang cukup lama,
yang akhirnya berdampak dalam penundaan pemberian materi mata kuliah Jaringan Komputer Lanjut itu sendiri. Penulis mengusulkan sebuah solusi yang dapat mempercepat
dan mempermudah proses peninjauan kompetensi mahasiswa Jaringan Komputer Lanjut melalui pengembangan sistem aplikasi web yang memberikan layanan lab
untuk mahasiswa, yang nantinya akan dimanfaatkan sebagai saran penilaian otomatis dan menggantikan evaluasi tes tertulis yang sebelumnya digunakan pada mata kuliah Jaringan Komputer Lanjut.
Sistem ini akan dibangun dengan memerhatikan aspek-aspek \textit{lab based learning} pada mahasiswa untuk memastikan bahwa hasil dari evaluasi yang dilakukan dengan penggunaan
lab merupakan hasil yang representatif terhadap kompetensi mahasiswa pada bidang tersebut.
\par

Dalam sistem aplikasi web ini, semua mahasiswa Jaringan Komputer Lanjut akan diberikan sebuah akun yang nantinya akan digunakan untuk mengakses aplikasi web.
Aplikasi web akan menyediakan beberapa lab yang secara keseluruhan akan mencakup seluruh materi Jaringan Komputer. Dalam sistem ini, lab yang diberikan berbentuk
mesin virtual yang dapat diakses oleh mahasiswa melalui \textit{web browser}. Mesin virtual ini telah dikonfigurasikan berdasarkan kebutuhan masing-masing topik lab
oleh dosen melalui sebuah halaman perantara pada sistem aplikasi web, dan mesin virtual ini berada di dalam sebuah jaringan virtual sehingga mesin virtual
seolah-olah berada dalam sebuah jaringan fisik dengan mesin-mesin lainnya. Mahasiswa akan diberikan sebuah halaman pada aplikasi web dimana mahasiswa dapat melihat
dan memilih seluruh lab yang tersedia. Ketika mahasiswa memutuskan untuk melaksanakan suatu lab tertentu, maka sistem akan memberikan informasi terkait tujuan dan
tugas yang perlu dilakukan oleh mahasiswa pada lab tersebut, kemudian sistem akan menyiapkan sebuah mesin virtual baru yang nantinya akan ditampilkan kepada
mahasiswa dalam \textit{web browser} mahasiswa. Mahasiswa memiliki keleluasaan untuk berinteraksi dengan mesin virtual layaknya mahasiswa berinteraksi dengan sebuah mesin fisik. 
\par

Setiap interaksi atau perintah yang dilakukan oleh mahasiswa pada mesin virtual akan dicatat oleh sistem untuk dievaluasi secara otomatis setelah lab selesai dilaksanakan.
Dosen dapat mengkonfigurasikan kondisi-kondisi yang digunakan oleh sistem untuk penilaian otomatis. Kondisi-kondisi yang dapat dikonfigurasi meliputi hal-hal seperti mengecek
apakah mahasiswa memberikan perintah tertentu dalam sesi lab tersebut, mengecek apakah koneksi antara dua \textit{node} pada jaringan virtual berhasil dibuat, dan lain-lain. 
Sistem ini memastikan bahwa setiap mahasiswa dapat mengimplementasikan konsep-konsep yang dipelajari pada mata kuliah Jaringan Komputer pada sebuah mesin yang terhubung pada sebuah jaringan,
dan sistem ini juga memastikan bahwa implementasi yang dilakukan oleh mahasiswa memang sesuai dengan implementasi yang diharapkan oleh dosen dengan memanfaatkan pengecekan kondisi-kondisi
yang telah ditetapkan sebelumnya. Dalam melaksanakan suatu lab, mahasiswa akan diberikan waktu pengerjaan yang dapat dikonfigurasi oleh dosen. Waktu pengerjaan akan dihitung ketika mahasiswa
memulai lab, dan lab akan dianggap selesai serta akan dievaluasi setelah mahasiswa menekan tombol khusus untuk menghentikan lab atau setelah waktu pengerjaan telah selesai.
\par




% Seiring berkembangnya teknologi informasi, pertukaran informasi antar komputer menjadi sebuah permasalahan yang semakin krusial. 
% Setiap orang yang berada di dalam dunia teknologi informasi pasti tidak akan terlepas dengan jaringan komputer, dan pemahaman 
% terhadap jaringan komputer sudah menjadi suatu hal yang diekspektasikan dari siapapun yang bekerja di dunia teknologi informasi. 
% Dalam hal ini, Fakultas Ilmu Komputer telah menyediakan mata kuliah Jaringan Komputer Lanjut untuk mengembangkan pemahaman mahasiswa 
% Fakultas Ilmu Komputer terhadap jaringan komputer dengan lebih dalam. Tentunya, mahasiswa juga perlu memiliki pemahaman yang mendasar 
% terhadap jaringan komputer untuk bisa melaksanakan pembelajaran dalam mata kuliah Jaringan Komputer Lanjut secara efektif. \par

% Dari kondisi tersebut, memastikan bahwa setiap mahasiswa memiliki kompetensi yang mencukupi untuk mendalami permasalahan jaringan komputer 
% merupakan sebuah tantangan yang perlu ditangani. Pada saat proses penulisan, solusi dari tantangan tersebut merupakan pengadaan materi
% \textit{review} oleh dosen Jaringan Komputer Lanjut yang dilaksanakan melalui pemberian materi ulang serta penilaian dalam bentuk kuis dan tes tertulis.
% Proses ini membutuhkan verifikasi manual yang memakan waktu satu minggu yang seharusnya dapat dimanfaatkan untuk mendalami materi-materi yang perlu dibahas
% dalam mata kuliah Jaringan Komputer Lanjut. Untuk mempermudah dan mempercepat proses yang memakan waktu ini, dapat dikembangkan sebuah 
% sistem aplikasi web dimana proses evaluasi kompetensi dilakukan melalui pengerjaan \textit{lab online} yang memiliki sistem penilaian otomatis seperti \textit{grader}.
% Lab dalam sistem ini menghubungkan mahasiswa dengan sebuah mesin virtual yang terhubung kedalam sebuah jaringan 
% virtual atau \textit{virtual network}, dan setiap perbuatan mahasiswa dalam mesin virtual akan dicatat dan dinilai melalui pengecekan log dari perintah yang telah diberikan
% oleh mahasiswa pada lab tersebut. 
% Tentunya, diperlukan sebuah sistem \textit{instance} dan partisi \textit{resource} yang memungkinkan sistem tersebut untuk digunakan dengan jumlah pengguna yang banyak.
% Setiap mahasiswa yang ingin mengerjakan sebuah lab akan diberikan sebuah \textit{instance} mesin virtual unik yang disediakan oleh sistem berdasarkan suatu \textit{template} yang telah didefinisikan sebelumnya, 
% sehingga setiap lingkungan lab tidak memiliki keterhubungan satu sama lain.
% \par

% Terdapat berbagai macam solusi sistem yang dapat dipilih dan dikembangkan untuk memenuhi kebutuhan-kebutuhan tersebut. Salah satu konsep yang dapat diterapkan
% adalah \textit{virtual network emulation}. 

% Terdapat banyak solusi yang dapat diterapkan agar sistem yang diinginkan dapat terealisasi. Konsep yang dipilih sebagai solusi dari permasalahan ini adalah konsep Network
% Virtualization. Untuk memahami apa itu \textit{Network Virtualization}, kita perlu memahami konsep \textit{virtualization}.
% \textit{Virtualization} secara prinsip meliputi penggunaan sebuah \textit{software layer} yang mengenkapsulasi sebuah sistem operasi, sehingga
% \textit{software layer} tersebut dapat memberikan input, output, dan \textit{behavior} yang identik dengan apa yang diharapkan dari sebuah perangkat fisik.
% \footnote{Michael Pearce, Sherali Zeadally, and Ray Hunt, “Virtualization,” ACM Computing Surveys 45, no. 2 (February 2013): 1–39, https://doi.org/10.1145/2431211.2431216.}
% \textit{Virtualization} memberikan sebuah abstraksi antara pengguna dan sumber daya fisik, sehingga pengguna seolah-olah dapat berinteraksi secara langsung dengan
% sumber daya fisik tersebut tanpa adanya hubungan atau koneksi langsung antara pengguna dan sumber daya fisik.
% \footnote{Carapinha, Jorge, and Javier Jiménez. “Network Virtualization: A View from the Bottom.” Proceedings of the 1st ACM workshop on Virtualized infrastructure systems and architectures (2009). https://doi.org/10.1145/1592648.1592660. }
% Berangkat dari pemahaman konsep \textit{virtualization} tersebut, Network Virtualization merupakan sebuah teknologi yang memungkinkan pengoperasian dari 
% banyak \textit{logical network} dalam satu perangkat fisik.
% \footnote{Tutschku, Kurt, Thomas Zinner, Akihiro Nakao, and Phuoc Tran-Gia. "Network virtualization: Implementation steps towards the future internet." Electronic Communications of the EASST 17 (2009).}
% Dengan memungkinkan adanya banyak arsitektur jaringan komputer yang bersifat heterogen dalam satu perangkat fisik, Network Virtualization memberikan flexibilitas, mempromosikan keragaman, menjamin keamanan, 
% dan meningkatkan pengelolaan.
% \footnote{N.M.M.K. Chowdhury and R. Boutaba, “Network Virtualization: State of the Art and Research Challenges,” IEEE Communications Magazine 47, no. 7 (July 2009): 20–26, https://doi.org/10.1109/mcom.2009.5183468.}
% \par

% Implementasi dari konsep atau teknologi Network Virtualization dilakukan dengan memanfaatkan Mininet. Mininet adalah sebuah \textit{network emulator} yang 
% dapat mengemulasikan sebuah jaringan virtual yang terdiri dari \textit{virtual hosts, switches, controllers,} dan \textit{links}. 
% \footnote{Mininet Project Contributors, “Mininet Overview,” Mininet, accessed February 9, 2024, https://mininet.org/overview/.}
% Mininet dipilih sebagai solusi dari \textit{emulator} jaringan virtual karena Mininet menyediakan sebuah API untuk Python yang dapat digunakan
% sebagai dasar dari \textit{backend} sistem web yang akan dikembangkan. Dengan Mininet, pembuatan sebuah jaringan virtual dapat dilaksanakan
% melalui \textit{command} yang telah disediakan oleh Mininet, dan interaksi dengan \textit{node} pada jaringan virtual tersebut dapat dilakukan
% dengan memanfaatkan API untuk Python. 

% Virtual Network itu sendiri secara umum merupakan sebuah teknologi atau konsep yang menghubungkan mesin dan/atau perangkat virtual 
% menggunakan perangkat lunak. 
% \footnote{What is virtual networking?, accessed February 9, 2024, https://www.vmware.com/topics/glossary/content/virtual-networking.html.} 
% Virtual Network memungkinkan mesin virtual untuk berkomunikasi dengan jaringan, \textit{host machine}, dan mesin virtual 
% lainnya tanpa adanya perangkat keras sebagai penghubung mesin-mesin tersebut layaknya dalam sebuah jaringan komputer tradisional. 
% \footnote{Thomas Olzak, Jason Boomer, Robert M. Keefer, James Sabovik. "Managing Hyper-V" Microsoft Virtualization (2010): 39-60. https://doi.org/10.1016/B978-1-59749-431-1.00004-7} 
% Virtual Network merupakan dasar dari sistem yang akan dikembangkan nantinya serta merupakan tujuan akhir dari pengimplementasian teknologi \textit{Network Virtualization},
% dimana setiap lab akan berada dalam sebuah lingkungan berupa Virtual Network, sehingga setiap \textit{instance} dari lab bersifat unik dan terisolasi satu sama lain.
% \par

%-----------------------------------------------------------------------------%
\section{Permasalahan}
\label{sec:masalah}
%-----------------------------------------------------------------------------%
Terdapat banyak aspek yang perlu diperhatikan dalam mengembangkan sebuah sistem yang telah disebutkan pada bagian sebelumnya. Setiap fitur yang nantinya akan dikembangkan
memiliki pilihan solusi komponen yang beragam, sehingga perlu ada sebuah analisis terhadap komponen apa saja yang akan dipilih sebagai solusi, dan bagaimana 
setiap komponen yang dipilih nantinya dapat berinteraksi dengan komponen-komponen lainnya sehingga sistem aplikasi web dapat berjalan dengan kohesif. 
\par

Selain itu, sistem yang dikembangkan perlu memiliki kemampuan untuk melayani banyak pengguna dalam suatu waktu tertentu. Untuk bisa menjalankan sebuah mesin virtual,
komputer yang berperan sebagai \textit{server} dari aplikasi web perlu mengalokasikan \textit{resource} agar mesin virtual bisa dijalankan. Apabila pengguna tidak berjumlah banyak,
hal ini bukanlah menjadi sebuah masalah. Namun, tentunya terdapat puluhan mahasiswa yang mengikuti mata kuliah Jaringan Komputer Lanjut, dan apabila setiap mahasiswa
mata kuliah Jaringan Komputer Lanjut ingin melaksanakan pengerjaan lab di waktu yang bersamaan, maka alokasi \textit{resource} dapat menjadi suatu masalah yang besar.
Apabila alokasi \textit{resource} tidak dapat dilakukan secara efektif, maka sistem beresiko untuk mengalami situasi \textit{down} yang dapat mengganggu proses pengerjaan lab.
Perlu dikembangkan sebuah solusi alokasi \textit{resource} yang efektif dan efisien, sehingga setiap mahasiswa dapat melaksanakan proses pengerjaan lab secara lancar.
\par

Di sisi lain, sebagai sebuah sistem yang ditujukan untuk menunjang pembelajaran dan pengujian kompetensi mahasiswa, perlu diperhatikan metodologi pembelajaran
yang akan dipersembahkan oleh sistem kepada mahasiswa. Untuk memastikan bahwa sistem pengujian berbasis lab dapat memastikan bahwa mahasiswa memiliki kompetensi yang memadai,
diperlukan adanya sebuah evaluasi terhadap sistem pengujian berbasis lab, analisis implementasi dari pengujian berbasis lab secara empiris, serta aspek-aspek lainnya
yang perlu dipertimbangkan dalam mengimplementasikan sebuah sistem pembelajaran atau pengujian berbasis lab.  
\par
%-----------------------------------------------------------------------------%
\subsection{Definisi Permasalahan}
\label{sec:definisiMasalah}
%-----------------------------------------------------------------------------%
Berdasarkan penjelasan pada bagian permasalahan, maka dapat didefinisikan rumusan permasalahan dari proyek ini sebagai berikut:
\begin{itemize}
	\item Bagaimana cara mengimplementasi virtual lab sebagai sistem evaluasi otomatis?
	\item Bagaimana cara mengoptimasi sistem dalam pembagian, penggunaan, dan alokasi resource yang terbatas?
	\item Bagaimana implementasi sistem lab agar pembelajaran dan pengujian kompetensi mahasiswa dapat dilaksanakan secara efektif?
\end{itemize}
-----------------------------------------------------------------------------%
\subsection{Batasan Permasalahan}
\label{sec:batasanMasalah}
%-----------------------------------------------------------------------------%
Dalam penelitian serta pengembangan aplikasi dari proyek ini, terdapat beberapa asumsi dan batasan permasalahan yang perlu didefinisikan, yaitu:
\begin{itemize}
	\item Sistem aplikasi web hanya akan digunakan dan dimanfaatkan oleh mahasiswa dan dosen Jaringan Komputer Lanjut sebagai sarana evaluasi.
	\item Pengembangan proyek, termasuk implementasi \textit{backend} dan \textit{frontend}, dilaksanakan secara efektif dalam rentang waktu 5 bulan antara bulan Februari 2024 hingga Juni 2024.
	\item Implementasi dari sebuah sistem evaluasi untuk mahasiswa Jaringan Komputer Lanjut berdasarkan konsep lab untuk mesin virtual yang dapat diakses oleh mahasiswa.
	\item Proyek tidak mengembangkan sistem jaringan virtual ataupun sistem mesin virtual dari awal, melainkan mengimplementasi dan mengintegrasikan layanan-layanan yang menyediakan jaringan virtual dan mesin virtual kedalam suatu aplikasi web.
\end{itemize}

%-----------------------------------------------------------------------------%
\section{Tujuan Penelitian}
\label{sec:tujuan}
%-----------------------------------------------------------------------------%
Berikut ini adalah tujuan penelitian yang dilakukan:
\begin{itemize}
	\item Mengembangkan sistem aplikasi web yang menyediakan layanan dalam bentuk lab virtual.
	\item Memudahkan proses evaluasi kompetensi mahasiswa Jaringan Komputer Lanjut.
\end{itemize}


%-----------------------------------------------------------------------------%
\section{Posisi Penelitian}
\label{sec:posisiPenelitian}
%-----------------------------------------------------------------------------%
\todo{
	Sebutkan posisi penelitian Anda. Ada baiknya jika Anda menggunakan gambar atau diagram.
	Template ini telah menyediakan contoh cara memasukkan gambar.
	}

\begin{figure}
	\centering
	\includegraphics[width=0.4\textwidth]{assets/pics/makara.png}
	\caption{Penjelasan singkat terkait gambar.}
	\label{fig:research_position}
\end{figure}

\noindent\todo{Jelaskan \pic~\ref{fig:research_position} di sini.}


% %-----------------------------------------------------------------------------%
\section{Langkah Penelitian (Metode Pemecahan Masalah)}
\label{sec:langkahPenelitian}
%-----------------------------------------------------------------------------%
Untuk memecahkan permasalahan yang telah dirumuskan, maka metode-metode berikut ini akan diterapkan dalam penelitian:
\begin{enumerate}
	\item Pendalaman materi terhadap Virtualization, Network Virtualization, Virtual Network, Virtual Machine, dan Operating System Logging. \\
	Sebelum pengembangan aplikasi dapat dilaksanakan, perlu adanya pemahaman yang mendalam terhadap konsep-konsep yang akan diterapkan sebagai sistem solusi dari permasalahan.
	\item Penentuan infrastruktur solusi \\
	Setelah pendalaman materi telah dilaksanakan, perlu dirumuskan sebuah solusi konkrit yang mencakup infrastruktur dan stack yang akan digunakan untuk pengembangan sistem.
	\item Pengembangan sistem \\
	Apabila infrastruktur dari sistem solusi telah ditentukan, maka pengembangan aplikasi berdasarkan perancangan yang telah dibuat dapat dilaksanakan.
\end{enumerate}


%-----------------------------------------------------------------------------%
\section{Sistematika Penulisan}
\label{sec:sistematikaPenulisan}
%-----------------------------------------------------------------------------%
Sistematika penulisan laporan adalah sebagai berikut:
\begin{itemize}
	\item Bab 1 \babSatu \\
	    Bab ini mencakup latar belakang, cakupan penelitian, dan pendefinisian masalah.
	\item Bab 2 \babDua \\
	    Bab ini mencakup pemaparan terminologi dan teori yang terkait dengan penelitian berdasarkan hasil tinjauan pustaka yang telah digunakan, sekaligus memperlihatkan kaitan teori dengan penelitian.
	\item Bab 3 \babTiga \\
	    Apa itu Bab 3?
	\item Bab 4 \babEmpat \\
		Apa itu Bab 4?
	\item Bab 5 \babLima \\
	    Apa itu Bab 5?
	\item Bab 6 \kesimpulan \\
	    Bab ini mencakup kesimpulan akhir penelitian dan saran untuk pengembangan berikutnya.
\end{itemize}

\noindent\todo{Anda bisa mengubah atau menambahkan penjelasan singkat mengenai isi masing-masing bab. Setiap tugas akhir pasti ada yang berbeda pada bagian ini.}

\clearchapter
% %-----------------------------------------------------------------------------%
\chapter{\babDua}
\label{bab:2}
%-----------------------------------------------------------------------------%
Terdapat banyak solusi yang dapat digunakan untuk mengimplementasikan sebuah sistem yang telah dipaparkan sebelumnya. Untuk bentuk sistem sendiri,
sistem yang akan dikembangkan nantinya akan berbentuk sebuah aplikasi web. Bentuk aplikasi web dipilih sebagai solusi karena sebuah aplikasi web dapat 
memberikan aspek efisiensi yang diinginkan, dimana sebuah aplikasi web dapat diakses oleh mahasiswa dimana saja dan kapan saja, sehingga proses
evaluasi juga dapat dilaksanakan tanpa memandang waktu ataupun tempat. Selain itu, sebuah sistem aplikasi web memungkinkan proses evaluasi untuk
di \textit{update} dan diobservasi oleh dosen secara \textit{realtime} dari mana saja. Hal ini mempermudah proses evaluasi bagi dosen dan mahasiswa, 
serta mempercepat proses pengecekan atau verifikasi yang dilakukan oleh dosen terhadap evaluasi kompetensi mahasiswa.
\par

Beranjak dari bentuk sistem tersebut, sistem yang nantinya akan dikembangkan merupakan sebuah aplikasi web yang memberikan layanan lab yang dapat 
diakses dan dikerjakan oleh mahasiswa, dimana lab ini nantinya akan memanfaatkan sistem penilaian otomatis untuk menilai kompetensi mahasiswa dalam 
pelaksanaan setiap lab yang diberikan. Sistem akan dibangun berdasarkan aspek-aspek \textit{virtual lab based learning}. Proses pengerjaan lab yang 
dinilai secara otomatis oleh sistem ini nantinya akan menggantikan proses evaluasi kompetensi manual yang saat ini digunakan dalam mata kuliah Jaringan Komputer Lanjut.
\par

Dalam sistem aplikasi web ini, semua mahasiswa Jaringan Komputer Lanjut akan diberikan sebuah akun yang nantinya akan digunakan untuk mengakses aplikasi web.
Aplikasi web akan menyediakan beberapa lab yang secara keseluruhan akan mencakup seluruh materi Jaringan Komputer. Dalam sistem ini, lab yang diberikan berbentuk
mesin virtual yang dapat diakses oleh mahasiswa melalui \textit{web browser}. Mesin virtual ini telah dikonfigurasikan berdasarkan kebutuhan masing-masing topik lab
oleh dosen melalui sebuah halaman perantara pada sistem aplikasi web, dan mesin virtual ini berada di dalam sebuah jaringan virtual sehingga mesin virtual
seolah-olah berada dalam sebuah jaringan fisik dengan mesin-mesin virtual lainnya. Mahasiswa akan diberikan sebuah halaman pada aplikasi web dimana mahasiswa dapat melihat
dan memilih seluruh lab yang tersedia. Ketika mahasiswa memutuskan untuk melaksanakan suatu lab tertentu, maka sistem akan memberikan informasi terkait tujuan dan
tugas yang perlu dilakukan oleh mahasiswa pada lab tersebut, kemudian sistem akan menyiapkan sebuah mesin virtual baru yang nantinya akan ditampilkan kepada
mahasiswa dalam \textit{web browser} mahasiswa. Mahasiswa memiliki keleluasaan untuk berinteraksi dengan mesin virtual layaknya mahasiswa berinteraksi dengan sebuah mesin fisik. 
\par

Setiap interaksi atau perintah yang dilakukan oleh mahasiswa pada mesin virtual akan dicatat oleh sistem untuk dievaluasi secara otomatis setelah lab selesai dilaksanakan.
Dosen dapat mengkonfigurasikan kondisi-kondisi yang digunakan oleh sistem untuk penilaian otomatis. Kondisi-kondisi yang dapat dikonfigurasi meliputi hal-hal seperti mengecek
apakah mahasiswa memberikan perintah tertentu dalam sesi lab tersebut, mengecek apakah koneksi antara dua \textit{node} pada jaringan virtual berhasil dibuat, dan lain-lain. 
Sistem ini memastikan bahwa setiap mahasiswa dapat mengimplementasikan konsep-konsep yang dipelajari pada mata kuliah Jaringan Komputer pada sebuah mesin yang terhubung pada sebuah jaringan,
dan sistem ini juga memastikan bahwa implementasi yang dilakukan oleh mahasiswa memang sesuai dengan implementasi yang diharapkan oleh dosen dengan memanfaatkan pengecekan kondisi-kondisi
yang telah ditetapkan sebelumnya. Dalam melaksanakan suatu lab, mahasiswa akan diberikan waktu pengerjaan yang dapat dikonfigurasi oleh dosen. Waktu pengerjaan akan dihitung ketika mahasiswa
memulai lab, dan lab akan dianggap selesai serta akan dievaluasi setelah mahasiswa menekan tombol khusus untuk menghentikan lab atau setelah waktu pengerjaan telah selesai.
\par

Untuk mengimplementasikan sistem tersebut, terdapat beberapa konsep dan teknologi yang akan digunakan sebagai fondasi dari sistem, diantaranya \textit{Network Virtualization},
\textit{Virtual Lab Based Learning}, dan lain-lain.
%-----------------------------------------------------------------------------%
\section{Network Virtualization dan Virtual Network}
\label{sec:latex}
%-----------------------------------------------------------------------------%
Untuk memahami apa itu \textit{Network Virtualization}, kita perlu memahami konsep \textit{virtualization}.
\textit{Virtualization} secara prinsip meliputi penggunaan sebuah \textit{software layer} yang mengenkapsulasi sebuah sistem operasi, sehingga
\textit{software layer} tersebut dapat memberikan input, output, dan \textit{behavior} yang identik dengan apa yang diharapkan dari sebuah perangkat fisik.
\footnote{Michael Pearce, Sherali Zeadally, and Ray Hunt, “Virtualization,” ACM Computing Surveys 45, no. 2 (February 2013): 1–39, https://doi.org/10.1145/2431211.2431216.}
\textit{Virtualization} memberikan sebuah abstraksi antara pengguna dan sumber daya fisik, sehingga pengguna seolah-olah dapat berinteraksi secara langsung dengan
sumber daya fisik tersebut tanpa adanya hubungan atau koneksi langsung antara pengguna dan sumber daya fisik.
\footnote{Carapinha, Jorge, and Javier Jiménez. “Network Virtualization: A View from the Bottom.” Proceedings of the 1st ACM workshop on Virtualized infrastructure systems and architectures (2009). https://doi.org/10.1145/1592648.1592660. }
% \citep{journal:nv_view_from_the_bottom}
Berangkat dari pemahaman konsep \textit{virtualization} tersebut, Network Virtualization merupakan sebuah teknologi yang memungkinkan pengoperasian dari 
banyak \textit{logical network} dalam satu perangkat fisik.
\footnote{Tutschku, Kurt, Thomas Zinner, Akihiro Nakao, and Phuoc Tran-Gia. "Network virtualization: Implementation steps towards the future internet." Electronic Communications of the EASST 17 (2009).}
Dengan memungkinkan adanya banyak arsitektur jaringan komputer yang bersifat heterogen dalam satu perangkat fisik, Network Virtualization memberikan flexibilitas, mempromosikan keragaman, menjamin keamanan, 
dan meningkatkan pengelolaan.
\footnote{N.M.M.K. Chowdhury and R. Boutaba, “Network Virtualization: State of the Art and Research Challenges,” IEEE Communications Magazine 47, no. 7 (July 2009): 20–26, https://doi.org/10.1109/mcom.2009.5183468.}
\par

Implementasi dari konsep atau teknologi Network Virtualization dilakukan dengan memanfaatkan Mininet. Mininet adalah sebuah \textit{network emulator} yang 
dapat mengemulasikan sebuah jaringan virtual yang terdiri dari \textit{virtual hosts, switches, controllers,} dan \textit{links}. 
\footnote{Mininet Project Contributors, “Mininet Overview,” Mininet, accessed February 9, 2024, https://mininet.org/overview/.}
Mininet dipilih sebagai solusi dari \textit{emulator} jaringan virtual karena Mininet menyediakan sebuah API untuk Python yang dapat digunakan
sebagai dasar dari \textit{backend} sistem web yang akan dikembangkan. 
\footnote{\textit{Ibid.}}
Dengan Mininet, pembuatan sebuah jaringan virtual dapat dilaksanakan melalui \textit{command} yang telah disediakan oleh Mininet, dan interaksi dengan \textit{node} 
pada jaringan virtual tersebut dapat dilakukan dengan memanfaatkan API untuk Python.
\footnote{\textit{Ibid.}}
\par

Virtual Network itu sendiri secara umum merupakan sebuah teknologi atau konsep yang menghubungkan mesin dan/atau perangkat virtual 
menggunakan perangkat lunak. 
\footnote{What is virtual networking?, accessed February 9, 2024, https://www.vmware.com/topics/glossary/content/virtual-networking.html.} 
Virtual Network memungkinkan mesin virtual untuk berkomunikasi dengan jaringan, \textit{host machine}, dan mesin virtual 
lainnya tanpa adanya perangkat keras sebagai penghubung mesin-mesin tersebut layaknya dalam sebuah jaringan komputer tradisional. 
\footnote{Thomas Olzak, Jason Boomer, Robert M. Keefer, James Sabovik. "Managing Hyper-V" Microsoft Virtualization (2010): 39-60. https://doi.org/10.1016/B978-1-59749-431-1.00004-7} 
Virtual Network merupakan dasar dari sistem yang akan dikembangkan nantinya serta merupakan tujuan akhir dari pengimplementasian teknologi \textit{Network Virtualization},
dimana setiap lab akan berada dalam sebuah lingkungan berupa Virtual Network, sehingga setiap \textit{instance} dari lab bersifat unik dan terisolasi satu sama lain.
\par
%-----------------------------------------------------------------------------%
\subsection{Virtual Lab Based Learning}
\label{sec:latexBrief}
%-----------------------------------------------------------------------------%
Lab yang bersifat \textit{hands on} dimana mahasiswa dapat berinteraksi dan menerapkan secara langsung ilmu yang mereka dapatkan merupakan suatu hal
yang penting untuk dilakukan.
\footnote{Le Xu, Dijiang Huang, and Wei-Tek Tsai, “V-Lab: A Cloud-Based Virtual Laboratory Platform for Hands-On Networking Courses,” Proceedings of the 17th ACM Annual Conference on Innovation and Technology in Computer Science Education, July 3, 2012, https://doi.org/10.1145/2325296.2325357.}
Sebuah lingkungan lab yang bagus perlu bersifat fleksibel dan dapat dikonfigurasi untuk mensimulasikan berbagai macam tipe dan konfigurasi jaringan komputer
untuk bisa diexperimentasikan. 
\footnote{\textit{Ibid.}}
Namun, cukup sulit bagi sebuah organisasi untuk dapat terus termutakhir terhadap perkembangan teknologi yang kompleks.
\footnote{\textit{Ibid.}}
\par

Terdapat lima tipe lab yang bersifat \textit{hands on} yang diterapkan dalam berbagai kampus di dunia.
\begin{itemize}
	\item \textit{Physical Lab}, yaitu lab yang dibuat dengan menggunakan perlengkapan fisik seperti \textit{router, switch,} dan lain-lain. Sistem lab ini dapat mensimulasikan permasalahan-permasalahan yang dihadapi dalam dunia nyata,
	tetapi disaat bersamaan lab yang berbentuk fisik lebih sulit untuk dikonfigurasi dan di-rekonfigurasi ketika terjadi perubahan \textit{requirement}. Selain itu, diperlukan juga staf yang bisa mengelola sistem tersebut, dan mahasiswa
	juga perlu mengikuti jadwal yang dibuat oleh tim pengajar.
	\item \textit{Virtual Application Laboratory}, yaitu lab yang memberikan akses gratis terhadap perangkat lunak berbayar yang tergolong mahal untuk mahasiswa melalui koneksi internet. Bentuk \textit{virtualization} ini terbatas terhadap
	\textit{application layer} sehingga sistem ini tidak bisa memberikan simulasi untuk pengoperasian level sistem operasi. Selain itu, sistem ini bersifat \textit{stateless}, sehingga progres akan hilang apabila sesi telah berakhir.
	\item \textit{Shared-Host Lab}, yaitu \textit{virtualization} dimana mahasiswa akan menggunakan perangkat lunak seperti RDP atau VNC untuk melakukan login kepada sistem lab. Oleh karena itu, \textit{host} dapat digunakan secara bersamaan 
	oleh lebih dari satu mahasiswa. Tentunya sistem ini tidak cocok untuk mengevaluasi kompetensi mahasiswa, karena sistem ini tidak bisa membedakan perintah mana yang dilakukan oleh mahasiswa tertentu.
	\item \textit{Single-VM Lab}, yaitu sistem dimana masing-masing pengguna akan mendapatkan satu \textit{virtual machine} yang berdasarkan sebuah \textit{template virtual machine}. Lab ini bersifat \textit{reservation based} dan \textit{stateless}.
	Namun, sistem ini tidak bisa dikonfigurasikan berdasarkan \textit{user requirement}, sehingga sistem ini menjadi sulit untuk diimplementasikan dalam kondisi dimana kurikulum lab sering berganti.
	\item \textit{Multi-VM \& Virtual Network Lab}, yaitu sistem yang memberikan \textit{virtualization} dengan lebih dari satu \textit{virtual machine} dan \textit{virtual network} yang didasarkan pada kebutuhan \textit{user}.
	\footnote{\textit{Ibid.}}
	Sistem aplikasi web yang akan dikembangkan bersifat mirip dengan sistem \textit{Multi-VM \& Virtual Network Lab}, dimana Mininet akan digunakan untuk sistem \textit{virtual network} yang bisa memiliki lebih dari satu \textit{host} yang berperan sebagai
	\textit{virtual machine} dalam \textit{virtual network} tersebut.  
\end{itemize}
%-----------------------------------------------------------------------------%
\subsection{\latex~Kompiler dan IDE}
\label{sec:latexCompiler}
%-----------------------------------------------------------------------------%
Untuk menggunakan \latex~(pada konteks hanya sebagai pengguna), tidak perlu banyak tahu mengenai hal-hal didalamnya.
Dengan menggunakan \f{Integrated Development Environment} (IDE), penggunaan \latex~akan serupa dengan pembuatan dokumen secara visual, layaknya OpenOffice Writer atau Microsoft Word.
Orang-orang yang menggunakan \latex~relatif lebih teliti dan terstruktur mengenai cara penulisan yang dia gunakan, karena \latex~memaksa untuk seperti itu.

Untuk mencoba \latex, diperlukan kompiler dan IDE.
Bagi pengguna Microsoft Windows dan Mac OS, instalasi kompiler \latex~dapat menggunakan MikTeX (\url{https://miktex.org/download}).
Bagi pengguna Linux, instalasi kompiler \latex~dapat menggunakan Texlive ( \url{http://www.tug.org/texlive/}).
Distro-distro \f{mainstream} di Linux seperti Ubuntu biasanya telah menyediakan \f{package} \code{texlive} melalui \f{package manager}.
Apabila ingin melakukan instalasi Texlive melalui \f{package manager}, lakukan instalasi package \code{texlive-full} atau setidaknya \code{texlive-science} agar prasyarat \f{template} ini tersedia secara lengkap.

Beberapa text editor atau IDE yang dapat digunakan adalah sebagai berikut:
\begin{itemize}
	\item TeXstudio (\url{https://www.texstudio.org/}).
	\item TeXWorks (biasanya bawaan dari MikTeX).
	\item Texmaker (\url{http://www.xm1math.net/texmaker/}).
	\item Microsoft Visual Studio Code, dengan \f{plugin} LaTeX Workshop (\url{https://marketplace.visualstudio.com/items?itemName=James-Yu.latex-workshop}).
	Untuk menggunakan \f{plugin} tersebut, diperlukan instalasi MikTeX dan Perl.
	Alternatif lain untuk persyaratan tersebut adalah menggunakan \f{plugin} Remote - WSL jika memiliki distro Windows Subsystem for Linux (WSL) 2 yang sudah terpasang \code{texlive}.
\end{itemize}


%-----------------------------------------------------------------------------%
\section{Panduan Pengunaan Dasar \latex}
\label{sec:latexUsage}
%-----------------------------------------------------------------------------%

%-----------------------------------------------------------------------------%
\subsection{Bold, Italic, dan Underline}
\label{sec:latexBIU}
%-----------------------------------------------------------------------------%
Hal pertama yang mungkin ditanyakan adalah bagaimana membuat huruf tercetak tebal, miring, atau memiliki garis bawah.
Pada Texmaker, Anda bisa melakukan hal ini seperti halnya saat mengubah dokumen dengan OO Writer.
Namun jika tetap masih tertarik dengan cara lain, ini dia:

\begin{itemize}
	\item \bo{Bold} \\
	Gunakan perintah \code{\bslash{}textbf$\lbrace\rbrace$} atau
	\code{\bslash{}bo$\lbrace\rbrace$}.
	\item \f{Italic} \\
	Gunakan perintah \code{\bslash{}textit$\lbrace\rbrace$} atau
	\code{\bslash{}f$\lbrace\rbrace$}.
	\item \underline{Underline} \\
	Gunakan perintah \code{\bslash{}underline$\lbrace\rbrace$}.
	\item $\overline{Overline}$ \\
	Gunakan perintah \code{\bslash{}overline}.
	\item $^{superscript}$ \\
	Gunakan perintah \code{\bslash{}$\lbrace\rbrace$}.
	\item $_{subscript}$ \\
	Gunakan perintah \code{\bslash{}\_$\lbrace\rbrace$}.
\end{itemize}

Perintah \code{\bslash{}f} dan \code{\bslash{}bo} hanya dapat digunakan jika package \code{uithesis} digunakan.

%-----------------------------------------------------------------------------%
\subsection{Memasukan Gambar}
\label{sec:latexImage}
%-----------------------------------------------------------------------------%
Setiap gambar dapat diberikan caption dan diberikan label. Label dapat digunakan untuk menunjuk gambar tertentu.
Jika posisi gambar berubah, maka nomor gambar juga akan diubah secara otomatis.
Begitu juga dengan seluruh referensi yang menunjuk pada gambar tersebut.
Contoh sederhana adalah \pic~\ref{fig:testGambar}.
Silahkan lihat code \latex~dengan nama bab2.tex untuk melihat kode lengkapnya.
Harap diingat bahwa caption untuk gambar selalu terletak dibawah gambar.

\begin{figure}
	\centering
	\includegraphics[width=0.50\textwidth]
	{assets/pics/creative_commons.png}
	\caption{\license.}
	\label{fig:testGambar}
\end{figure}


%-----------------------------------------------------------------------------%
\section{Membuat Tabel}
\label{sec:latexTable}
%-----------------------------------------------------------------------------%
Tabel pada Latex dapat dibuat dengan bantuan \textit{website} seperti \url{https://www.tablesgenerator.com/}.
Dengan menggunakan \textit{website} ini, maka pembuatan tabel akan menjadi lebih mudah.
\textit{User interface} dari \url{https://www.tablesgenerator.com/} dapat dilihat pada Gambar \ref{fig:tablesgenerator}.

\begin{figure}
	\centering
	\includegraphics[width=0.5\textwidth]{assets/pics/tablesgenerator-dot-com.png}
	\caption{\textit{User interface} dari \textit{website} https://www.tablesgenerator.com/}
	\label{fig:tablesgenerator}
\end{figure}

Di sisi lain, tabel juga dapat diberi label dan caption seperti pada gambar.
Caption pada tabel terletak pada bagian atas tabel.
Contoh tabel sederhana dapat dilihat pada \tab~\ref{tab:tab1}.

\begin{table}
	\centering
	\caption{Contoh Tabel}
	\label{tab:tab1}
	\begin{tabular}{| l | c r |}
		\hline
		& kol 1 & kol 2 \\
		\hline
		baris 1 & 1 & 2 \\
		baris 2 & 3 & 4 \\
		baris 3 & 5 & 6 \\
		baris 4 & 7 & 8 \\
		baris 5 & 9 & 10 \\
		\hline
		jumlah  & 25 & 30 \\
		\hline
	\end{tabular}
\end{table}

Adapun untuk membuat tabel panjang yang bisa melebihi dari satu halaman, gunakan perintah \code{\bslash{}begin\{longtable\}} sebagai pengganti \code{\bslash{}begin\{table\}}. Di dalam \code{longtable} tidak perlu lagi ada \code{\bslash{}begin\{tabular\}}. Kemudian, tambahkan tanda \code{\bslash{}\bslash{}} setelah baris \code{\bslash{}label\{....\}}, agar tidak menimbulkan error saat menampilkan \f{caption} di bagian atas tabel. Kemudian, untuk membatasi header yang ingin diulang pada halaman-halaman berikutnya, gunakan perintah \code{\bslash{}endhead}. Contohnya adalah sebagai berikut:

\begin{longtable}{| l | c r |}
\caption{Contoh Tabel Panjang}
\label{tab:tab2} \\
\hline
& kol 1 & kol 2 \\
\hline
\endfirsthead % batas akhir header yang akan muncul di halaman pertama
\hline
& kol 1 & kol 2 \\
\hline
\endhead % batas akhir header yang akan muncul di halaman berikutnya
baris 1  & 1 & 2 \\
baris 2  & 3 & 4 \\
baris 3  & 5 & 6 \\
baris 4  & 7 & 8 \\
baris 5  & 9 & 10 \\
baris 6  & 11 & 12 \\
baris 7  & 13 & 14 \\
baris 8  & 15 & 16 \\
baris 9  & 17 & 18 \\
baris 10 & 19 & 20 \\
baris 11 & 21 & 22 \\
baris 12 & 23 & 24 \\
baris 13 & 25 & 26 \\
baris 14 & 27 & 28 \\
baris 15 & 29 & 30 \\
\hline
\end{longtable}

Ada jenis tabel lain yang dapat dibuat dengan \latex~berikut beberapa diantaranya.
Contoh-contoh ini bersumber dari \url{http://en.wikibooks.org/wiki/LaTeX/Tables}

\begin{table}
	\centering
	\caption{An Example of Rows Spanning Multiple Columns}
	\label{row.spanning}
	\begin{tabular}{|l|l|*{6}{c|}}
		\hline % create horizontal line
		No & Name & \multicolumn{3}{|c|}{Week 1} & \multicolumn{3}{|c|}{Week 2} \\
		\cline{3-8} % create line from 3rd column till 8th column
		& & A & B & C & A & B & C\\
		\hline
		1 & Lala & 1 & 2 & 3 & 4 & 5 & 6\\
		2 & Lili & 1 & 2 & 3 & 4 & 5 & 6\\
		3 & Lulu & 1 & 2 & 3 & 4 & 5 & 6\\
		\hline
	\end{tabular}
\end{table}

\begin{table}
	\centering
	\caption{An Example of Columns Spanning Multiple Rows}
	\label{column.spanning}
	\begin{tabular}{|l|c|l|}
		\hline
		Percobaan & Iterasi & Waktu \\
		\hline
		Pertama & 1 & 0.1 sec \\ \hline
		\multirow{2}{*}{Kedua} & 1 & 0.1 sec \\
		& 3 & 0.15 sec \\
		\hline
		\multirow{3}{*}{Ketiga} & 1 & 0.09 sec \\
		& 2 & 0.16 sec \\
		& 3 & 0.21 sec \\
		\hline
	\end{tabular}
\end{table}

\begin{table}
	\centering
	\caption{An Example of Spanning in Both Directions Simultaneously}
	\label{mix.spanning}
	\begin{tabular}{cc|c|c|c|c|}
		\cline{3-6}
		& & \multicolumn{4}{|c|}{Title} \\ \cline{3-6}
		& & A & B & C & D \\ \hline
		\multicolumn{1}{|c|}{\multirow{2}{*}{Type}} &
		\multicolumn{1}{|c|}{X} & 1 & 2 & 3 & 4\\ \cline{2-6}
		\multicolumn{1}{|c|}{}                        &
		\multicolumn{1}{|c|}{Y} & 0.5 & 1.0 & 1.5 & 2.0\\ \cline{1-6}
		\multicolumn{1}{|c|}{\multirow{2}{*}{Resource}} &
		\multicolumn{1}{|c|}{I} & 10 & 20 & 30 & 40\\ \cline{2-6}
		\multicolumn{1}{|c|}{}                        &
		\multicolumn{1}{|c|}{J} & 5 & 10 & 15 & 20\\ \cline{1-6}
	\end{tabular}
\end{table}


%-----------------------------------------------------------------------------%
\section{Keterkaitan Teori Dengan Penelitian}
\label{sec:keterkaitan}
%-----------------------------------------------------------------------------%
\todo{Ada baiknya setelah menjelaskan teori-teori, Anda menjelaskan apa kaitan teori tersebut dengan penelitian Anda.
Hal ini tentunya membantu pembaca dalam memahami bahwa teori yang Anda paparkan memang penting untuk memahami penelitian Anda nantinya.}

\begin{figure}
	\centering
	\includegraphics[width=\textwidth]{assets/pics/research_concept_map.png}
	\caption{Keterkaitan konsep hasil studi literatur terhadap penelitian}
	\label{fig:research_concept_map}
\end{figure}

\noindent\todo{
	Jelaskan \pic~\ref{fig:research_concept_map} di sini.
	Setiap gambar pada tugas akhir butuh penjelasan.
	Gambar hadir untuk mempermudah membaca memahami konteks, tetapi tidak bisa berdiri sendiri tanpa penjelasan.
	Terkait gambar, Anda juga bisa mengatur skalanya.
	Gambar kali ini lebarnya 0,8x dari lebar teks halaman.
}

% \clearchapter
% \include{src/01-body/bab3}
% \clearchapter
% \include{src/01-body/bab4}
% \clearchapter
% \include{src/01-body/bab5}
% \clearchapter
% \include{src/01-body/kesimpulan}
% \clearchapter

%
% Daftar Pustaka
% \CAPinToC % All entries in ToC will be CAPITALIZED from here on
% \include{_internals/pustaka}
% \clearchapter
% \noCAPinToC % Revert to original \addcontentsline formatting

%
% Lampiran
%
% \begin{appendix}
% 	\newcounter{pagetemp}
% 	\setcounter{pagetemp}{\thepage}
% 	\include{_internals/markLampiran}
% 	\clearchapter
% 	\setcounter{page}{\thepagetemp}
% 	\stepcounter{page}
% 	\include{src/99-backMatter/lampiran}
% \end{appendix}

\end{document}
