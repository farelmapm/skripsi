%-----------------------------------------------------------------------------%
\chapter{\babDua}
\label{bab:2}
%-----------------------------------------------------------------------------%
Terdapat banyak solusi yang dapat digunakan untuk mengimplementasikan sebuah sistem yang telah dipaparkan sebelumnya. Untuk bentuk sistem sendiri,
sistem yang akan dikembangkan nantinya akan berbentuk sebuah aplikasi web. Bentuk aplikasi web dipilih sebagai solusi karena sebuah aplikasi web dapat 
memberikan aspek efisiensi yang diinginkan, dimana sebuah aplikasi web dapat diakses oleh mahasiswa dimana saja dan kapan saja, sehingga proses
evaluasi juga dapat dilaksanakan tanpa memandang waktu ataupun tempat. Selain itu, sebuah sistem aplikasi web memungkinkan proses evaluasi untuk
di \textit{update} dan diobservasi oleh dosen secara \textit{realtime} dari mana saja. Hal ini mempermudah proses evaluasi bagi dosen dan mahasiswa, 
serta mempercepat proses pengecekan atau verifikasi yang dilakukan oleh dosen terhadap evaluasi kompetensi mahasiswa.
\par

Beranjak dari bentuk sistem tersebut, sistem yang nantinya akan dikembangkan merupakan sebuah aplikasi web yang memberikan layanan lab yang dapat 
diakses dan dikerjakan oleh mahasiswa, dimana lab ini nantinya akan memanfaatkan sistem penilaian otomatis untuk menilai kompetensi mahasiswa dalam 
pelaksanaan setiap lab yang diberikan. Sistem akan dibangun berdasarkan aspek-aspek \textit{virtual lab based learning}. Proses pengerjaan lab yang 
dinilai secara otomatis oleh sistem ini nantinya akan menggantikan proses evaluasi kompetensi manual yang saat ini digunakan dalam mata kuliah Jaringan Komputer Lanjut.
\par

Dalam sistem aplikasi web ini, semua mahasiswa Jaringan Komputer Lanjut akan diberikan sebuah akun yang nantinya akan digunakan untuk mengakses aplikasi web.
Aplikasi web akan menyediakan beberapa lab yang secara keseluruhan akan mencakup seluruh materi Jaringan Komputer. Dalam sistem ini, lab yang diberikan berbentuk
mesin virtual yang dapat diakses oleh mahasiswa melalui \textit{web browser}. Mesin virtual ini telah dikonfigurasikan berdasarkan kebutuhan masing-masing topik lab
oleh dosen melalui sebuah halaman perantara pada sistem aplikasi web, dan mesin virtual ini berada di dalam sebuah jaringan virtual sehingga mesin virtual
seolah-olah berada dalam sebuah jaringan fisik dengan mesin-mesin virtual lainnya. Mahasiswa akan diberikan sebuah halaman pada aplikasi web dimana mahasiswa dapat melihat
dan memilih seluruh lab yang tersedia. Ketika mahasiswa memutuskan untuk melaksanakan suatu lab tertentu, maka sistem akan memberikan informasi terkait tujuan dan
tugas yang perlu dilakukan oleh mahasiswa pada lab tersebut, kemudian sistem akan menyiapkan sebuah mesin virtual baru yang nantinya akan ditampilkan kepada
mahasiswa dalam \textit{web browser} mahasiswa. Mahasiswa memiliki keleluasaan untuk berinteraksi dengan mesin virtual layaknya mahasiswa berinteraksi dengan sebuah mesin fisik. 
\par

Setiap interaksi atau perintah yang dilakukan oleh mahasiswa pada mesin virtual akan dicatat oleh sistem untuk dievaluasi secara otomatis setelah lab selesai dilaksanakan.
Dosen dapat mengkonfigurasikan kondisi-kondisi yang digunakan oleh sistem untuk penilaian otomatis. Kondisi-kondisi yang dapat dikonfigurasi meliputi hal-hal seperti mengecek
apakah mahasiswa memberikan perintah tertentu dalam sesi lab tersebut, mengecek apakah koneksi antara dua \textit{node} pada jaringan virtual berhasil dibuat, dan lain-lain. 
Sistem ini memastikan bahwa setiap mahasiswa dapat mengimplementasikan konsep-konsep yang dipelajari pada mata kuliah Jaringan Komputer pada sebuah mesin yang terhubung pada sebuah jaringan,
dan sistem ini juga memastikan bahwa implementasi yang dilakukan oleh mahasiswa memang sesuai dengan implementasi yang diharapkan oleh dosen dengan memanfaatkan pengecekan kondisi-kondisi
yang telah ditetapkan sebelumnya. Dalam melaksanakan suatu lab, mahasiswa akan diberikan waktu pengerjaan yang dapat dikonfigurasi oleh dosen. Waktu pengerjaan akan dihitung ketika mahasiswa
memulai lab, dan lab akan dianggap selesai serta akan dievaluasi setelah mahasiswa menekan tombol khusus untuk menghentikan lab atau setelah waktu pengerjaan telah selesai.
\par

Untuk mengimplementasikan sistem tersebut, terdapat beberapa konsep dan teknologi yang akan digunakan sebagai fondasi dari sistem, diantaranya \textit{Network Virtualization},
\textit{Virtual Lab Based Learning}, dan lain-lain.
%-----------------------------------------------------------------------------%
\section{Network Virtualization dan Virtual Network}
\label{sec:latex}
%-----------------------------------------------------------------------------%
Untuk memahami apa itu \textit{Network Virtualization}, kita perlu memahami konsep \textit{virtualization}.
\textit{Virtualization} secara prinsip meliputi penggunaan sebuah \textit{software layer} yang mengenkapsulasi sebuah sistem operasi, sehingga
\textit{software layer} tersebut dapat memberikan input, output, dan \textit{behavior} yang identik dengan apa yang diharapkan dari sebuah perangkat fisik.
\footnote{Michael Pearce, Sherali Zeadally, and Ray Hunt, “Virtualization,” ACM Computing Surveys 45, no. 2 (February 2013): 1–39, https://doi.org/10.1145/2431211.2431216.}
\textit{Virtualization} memberikan sebuah abstraksi antara pengguna dan sumber daya fisik, sehingga pengguna seolah-olah dapat berinteraksi secara langsung dengan
sumber daya fisik tersebut tanpa adanya hubungan atau koneksi langsung antara pengguna dan sumber daya fisik.
\footnote{Carapinha, Jorge, and Javier Jiménez. “Network Virtualization: A View from the Bottom.” Proceedings of the 1st ACM workshop on Virtualized infrastructure systems and architectures (2009). https://doi.org/10.1145/1592648.1592660. }
Berangkat dari pemahaman konsep \textit{virtualization} tersebut, Network Virtualization merupakan sebuah teknologi yang memungkinkan pengoperasian dari 
banyak \textit{logical network} dalam satu perangkat fisik.
\footnote{Tutschku, Kurt, Thomas Zinner, Akihiro Nakao, and Phuoc Tran-Gia. "Network virtualization: Implementation steps towards the future internet." Electronic Communications of the EASST 17 (2009).}
Dengan memungkinkan adanya banyak arsitektur jaringan komputer yang bersifat heterogen dalam satu perangkat fisik, Network Virtualization memberikan flexibilitas, mempromosikan keragaman, menjamin keamanan, 
dan meningkatkan pengelolaan.
\footnote{N.M.M.K. Chowdhury and R. Boutaba, “Network Virtualization: State of the Art and Research Challenges,” IEEE Communications Magazine 47, no. 7 (July 2009): 20–26, https://doi.org/10.1109/mcom.2009.5183468.}
\par

Implementasi dari konsep atau teknologi Network Virtualization dilakukan dengan memanfaatkan Mininet. Mininet adalah sebuah \textit{network emulator} yang 
dapat mengemulasikan sebuah jaringan virtual yang terdiri dari \textit{virtual hosts, switches, controllers,} dan \textit{links}. 
\footnote{Mininet Project Contributors, “Mininet Overview,” Mininet, accessed February 9, 2024, https://mininet.org/overview/.}
Mininet dipilih sebagai solusi dari \textit{emulator} jaringan virtual karena Mininet menyediakan sebuah API untuk Python yang dapat digunakan
sebagai dasar dari \textit{backend} sistem web yang akan dikembangkan. 
\footnote{\textit{Ibid.}}
Dengan Mininet, pembuatan sebuah jaringan virtual dapat dilaksanakan melalui \textit{command} yang telah disediakan oleh Mininet, dan interaksi dengan \textit{node} 
pada jaringan virtual tersebut dapat dilakukan dengan memanfaatkan API untuk Python.
\footnote{\textit{Ibid.}}
\par

Virtual Network itu sendiri secara umum merupakan sebuah teknologi atau konsep yang menghubungkan mesin dan/atau perangkat virtual 
menggunakan perangkat lunak. 
\footnote{What is virtual networking?, accessed February 9, 2024, https://www.vmware.com/topics/glossary/content/virtual-networking.html.} 
Virtual Network memungkinkan mesin virtual untuk berkomunikasi dengan jaringan, \textit{host machine}, dan mesin virtual 
lainnya tanpa adanya perangkat keras sebagai penghubung mesin-mesin tersebut layaknya dalam sebuah jaringan komputer tradisional. 
\footnote{Thomas Olzak, Jason Boomer, Robert M. Keefer, James Sabovik. "Managing Hyper-V" Microsoft Virtualization (2010): 39-60. https://doi.org/10.1016/B978-1-59749-431-1.00004-7} 
Virtual Network merupakan dasar dari sistem yang akan dikembangkan nantinya serta merupakan tujuan akhir dari pengimplementasian teknologi \textit{Network Virtualization},
dimana setiap lab akan berada dalam sebuah lingkungan berupa Virtual Network, sehingga setiap \textit{instance} dari lab bersifat unik dan terisolasi satu sama lain.
\par
%-----------------------------------------------------------------------------%
\subsection{\latex~Secara Singkat}
\label{sec:latexBrief}
%-----------------------------------------------------------------------------%
Berdasarkan \cite{latex:intro}: \\
\begin{tabular}{| p{13cm} |}
	\hline
	\\
	LaTeX is a family of programs designed to produce publication-quality typeset documents.
	It is particularly strong when working with mathematical symbols. \\
	The history of LaTeX begins with a program called TEX.
	In 1978, a computer scientist by the name of Donald Knuth grew frustrated with the mistakes that his publishers made in typesetting his work.
	He decided to create a typesetting program that everyone could easily use to typeset documents, particularly those that include formulae, and made it freely available.
	The result is TEX. \\
	Knuth's product is an immensely powerful program, but one that does focus very much on small details.
	A mathematician and computer scientist by the name of Leslie Lamport wrote a variant of TEX called \LaTeX that focuses on document structure rather than such details. \\
	\\
	\hline
\end{tabular}

\vspace*{0.8cm}

Dokumen \LaTeX~sangat mudah, seperti halnya membuat dokumen teks biasa.
Ada beberapa perintah yang diawali dengan tanda '\bslash'.
Seperti perintah \code{\bslash\bslash}~yang digunakan untuk memberi baris baru.
Perintah tersebut juga sama dengan perintah \code{\bslash{}newline}.
Pada bagian ini akan sedikit dijelaskan cara manipulasi teks dan perintah-perintah \latex~yang mungkin akan sering digunakan.
Jika ingin belajar hal-hal dasar mengenai \latex, silakan kunjungi:

\begin{itemize}
	\item \url{http://frodo.elon.edu/tutorial/tutorial/}, atau
	\item \url{http://www.maths.tcd.ie/~dwilkins/LaTeXPrimer/}
\end{itemize}

%-----------------------------------------------------------------------------%
\subsection{\latex~Kompiler dan IDE}
\label{sec:latexCompiler}
%-----------------------------------------------------------------------------%
Untuk menggunakan \latex~(pada konteks hanya sebagai pengguna), tidak perlu banyak tahu mengenai hal-hal didalamnya.
Dengan menggunakan \f{Integrated Development Environment} (IDE), penggunaan \latex~akan serupa dengan pembuatan dokumen secara visual, layaknya OpenOffice Writer atau Microsoft Word.
Orang-orang yang menggunakan \latex~relatif lebih teliti dan terstruktur mengenai cara penulisan yang dia gunakan, karena \latex~memaksa untuk seperti itu.

Untuk mencoba \latex, diperlukan kompiler dan IDE.
Bagi pengguna Microsoft Windows dan Mac OS, instalasi kompiler \latex~dapat menggunakan MikTeX (\url{https://miktex.org/download}).
Bagi pengguna Linux, instalasi kompiler \latex~dapat menggunakan Texlive ( \url{http://www.tug.org/texlive/}).
Distro-distro \f{mainstream} di Linux seperti Ubuntu biasanya telah menyediakan \f{package} \code{texlive} melalui \f{package manager}.
Apabila ingin melakukan instalasi Texlive melalui \f{package manager}, lakukan instalasi package \code{texlive-full} atau setidaknya \code{texlive-science} agar prasyarat \f{template} ini tersedia secara lengkap.

Beberapa text editor atau IDE yang dapat digunakan adalah sebagai berikut:
\begin{itemize}
	\item TeXstudio (\url{https://www.texstudio.org/}).
	\item TeXWorks (biasanya bawaan dari MikTeX).
	\item Texmaker (\url{http://www.xm1math.net/texmaker/}).
	\item Microsoft Visual Studio Code, dengan \f{plugin} LaTeX Workshop (\url{https://marketplace.visualstudio.com/items?itemName=James-Yu.latex-workshop}).
	Untuk menggunakan \f{plugin} tersebut, diperlukan instalasi MikTeX dan Perl.
	Alternatif lain untuk persyaratan tersebut adalah menggunakan \f{plugin} Remote - WSL jika memiliki distro Windows Subsystem for Linux (WSL) 2 yang sudah terpasang \code{texlive}.
\end{itemize}


%-----------------------------------------------------------------------------%
\section{Panduan Pengunaan Dasar \latex}
\label{sec:latexUsage}
%-----------------------------------------------------------------------------%

%-----------------------------------------------------------------------------%
\subsection{Bold, Italic, dan Underline}
\label{sec:latexBIU}
%-----------------------------------------------------------------------------%
Hal pertama yang mungkin ditanyakan adalah bagaimana membuat huruf tercetak tebal, miring, atau memiliki garis bawah.
Pada Texmaker, Anda bisa melakukan hal ini seperti halnya saat mengubah dokumen dengan OO Writer.
Namun jika tetap masih tertarik dengan cara lain, ini dia:

\begin{itemize}
	\item \bo{Bold} \\
	Gunakan perintah \code{\bslash{}textbf$\lbrace\rbrace$} atau
	\code{\bslash{}bo$\lbrace\rbrace$}.
	\item \f{Italic} \\
	Gunakan perintah \code{\bslash{}textit$\lbrace\rbrace$} atau
	\code{\bslash{}f$\lbrace\rbrace$}.
	\item \underline{Underline} \\
	Gunakan perintah \code{\bslash{}underline$\lbrace\rbrace$}.
	\item $\overline{Overline}$ \\
	Gunakan perintah \code{\bslash{}overline}.
	\item $^{superscript}$ \\
	Gunakan perintah \code{\bslash{}$\lbrace\rbrace$}.
	\item $_{subscript}$ \\
	Gunakan perintah \code{\bslash{}\_$\lbrace\rbrace$}.
\end{itemize}

Perintah \code{\bslash{}f} dan \code{\bslash{}bo} hanya dapat digunakan jika package \code{uithesis} digunakan.

%-----------------------------------------------------------------------------%
\subsection{Memasukan Gambar}
\label{sec:latexImage}
%-----------------------------------------------------------------------------%
Setiap gambar dapat diberikan caption dan diberikan label. Label dapat digunakan untuk menunjuk gambar tertentu.
Jika posisi gambar berubah, maka nomor gambar juga akan diubah secara otomatis.
Begitu juga dengan seluruh referensi yang menunjuk pada gambar tersebut.
Contoh sederhana adalah \pic~\ref{fig:testGambar}.
Silahkan lihat code \latex~dengan nama bab2.tex untuk melihat kode lengkapnya.
Harap diingat bahwa caption untuk gambar selalu terletak dibawah gambar.

\begin{figure}
	\centering
	\includegraphics[width=0.50\textwidth]
	{assets/pics/creative_commons.png}
	\caption{\license.}
	\label{fig:testGambar}
\end{figure}


%-----------------------------------------------------------------------------%
\section{Membuat Tabel}
\label{sec:latexTable}
%-----------------------------------------------------------------------------%
Tabel pada Latex dapat dibuat dengan bantuan \textit{website} seperti \url{https://www.tablesgenerator.com/}.
Dengan menggunakan \textit{website} ini, maka pembuatan tabel akan menjadi lebih mudah.
\textit{User interface} dari \url{https://www.tablesgenerator.com/} dapat dilihat pada Gambar \ref{fig:tablesgenerator}.

\begin{figure}
	\centering
	\includegraphics[width=0.5\textwidth]{assets/pics/tablesgenerator-dot-com.png}
	\caption{\textit{User interface} dari \textit{website} https://www.tablesgenerator.com/}
	\label{fig:tablesgenerator}
\end{figure}

Di sisi lain, tabel juga dapat diberi label dan caption seperti pada gambar.
Caption pada tabel terletak pada bagian atas tabel.
Contoh tabel sederhana dapat dilihat pada \tab~\ref{tab:tab1}.

\begin{table}
	\centering
	\caption{Contoh Tabel}
	\label{tab:tab1}
	\begin{tabular}{| l | c r |}
		\hline
		& kol 1 & kol 2 \\
		\hline
		baris 1 & 1 & 2 \\
		baris 2 & 3 & 4 \\
		baris 3 & 5 & 6 \\
		baris 4 & 7 & 8 \\
		baris 5 & 9 & 10 \\
		\hline
		jumlah  & 25 & 30 \\
		\hline
	\end{tabular}
\end{table}

Adapun untuk membuat tabel panjang yang bisa melebihi dari satu halaman, gunakan perintah \code{\bslash{}begin\{longtable\}} sebagai pengganti \code{\bslash{}begin\{table\}}. Di dalam \code{longtable} tidak perlu lagi ada \code{\bslash{}begin\{tabular\}}. Kemudian, tambahkan tanda \code{\bslash{}\bslash{}} setelah baris \code{\bslash{}label\{....\}}, agar tidak menimbulkan error saat menampilkan \f{caption} di bagian atas tabel. Kemudian, untuk membatasi header yang ingin diulang pada halaman-halaman berikutnya, gunakan perintah \code{\bslash{}endhead}. Contohnya adalah sebagai berikut:

\begin{longtable}{| l | c r |}
\caption{Contoh Tabel Panjang}
\label{tab:tab2} \\
\hline
& kol 1 & kol 2 \\
\hline
\endfirsthead % batas akhir header yang akan muncul di halaman pertama
\hline
& kol 1 & kol 2 \\
\hline
\endhead % batas akhir header yang akan muncul di halaman berikutnya
baris 1  & 1 & 2 \\
baris 2  & 3 & 4 \\
baris 3  & 5 & 6 \\
baris 4  & 7 & 8 \\
baris 5  & 9 & 10 \\
baris 6  & 11 & 12 \\
baris 7  & 13 & 14 \\
baris 8  & 15 & 16 \\
baris 9  & 17 & 18 \\
baris 10 & 19 & 20 \\
baris 11 & 21 & 22 \\
baris 12 & 23 & 24 \\
baris 13 & 25 & 26 \\
baris 14 & 27 & 28 \\
baris 15 & 29 & 30 \\
\hline
\end{longtable}

Ada jenis tabel lain yang dapat dibuat dengan \latex~berikut beberapa diantaranya.
Contoh-contoh ini bersumber dari \url{http://en.wikibooks.org/wiki/LaTeX/Tables}

\begin{table}
	\centering
	\caption{An Example of Rows Spanning Multiple Columns}
	\label{row.spanning}
	\begin{tabular}{|l|l|*{6}{c|}}
		\hline % create horizontal line
		No & Name & \multicolumn{3}{|c|}{Week 1} & \multicolumn{3}{|c|}{Week 2} \\
		\cline{3-8} % create line from 3rd column till 8th column
		& & A & B & C & A & B & C\\
		\hline
		1 & Lala & 1 & 2 & 3 & 4 & 5 & 6\\
		2 & Lili & 1 & 2 & 3 & 4 & 5 & 6\\
		3 & Lulu & 1 & 2 & 3 & 4 & 5 & 6\\
		\hline
	\end{tabular}
\end{table}

\begin{table}
	\centering
	\caption{An Example of Columns Spanning Multiple Rows}
	\label{column.spanning}
	\begin{tabular}{|l|c|l|}
		\hline
		Percobaan & Iterasi & Waktu \\
		\hline
		Pertama & 1 & 0.1 sec \\ \hline
		\multirow{2}{*}{Kedua} & 1 & 0.1 sec \\
		& 3 & 0.15 sec \\
		\hline
		\multirow{3}{*}{Ketiga} & 1 & 0.09 sec \\
		& 2 & 0.16 sec \\
		& 3 & 0.21 sec \\
		\hline
	\end{tabular}
\end{table}

\begin{table}
	\centering
	\caption{An Example of Spanning in Both Directions Simultaneously}
	\label{mix.spanning}
	\begin{tabular}{cc|c|c|c|c|}
		\cline{3-6}
		& & \multicolumn{4}{|c|}{Title} \\ \cline{3-6}
		& & A & B & C & D \\ \hline
		\multicolumn{1}{|c|}{\multirow{2}{*}{Type}} &
		\multicolumn{1}{|c|}{X} & 1 & 2 & 3 & 4\\ \cline{2-6}
		\multicolumn{1}{|c|}{}                        &
		\multicolumn{1}{|c|}{Y} & 0.5 & 1.0 & 1.5 & 2.0\\ \cline{1-6}
		\multicolumn{1}{|c|}{\multirow{2}{*}{Resource}} &
		\multicolumn{1}{|c|}{I} & 10 & 20 & 30 & 40\\ \cline{2-6}
		\multicolumn{1}{|c|}{}                        &
		\multicolumn{1}{|c|}{J} & 5 & 10 & 15 & 20\\ \cline{1-6}
	\end{tabular}
\end{table}


%-----------------------------------------------------------------------------%
\section{Keterkaitan Teori Dengan Penelitian}
\label{sec:keterkaitan}
%-----------------------------------------------------------------------------%
\todo{Ada baiknya setelah menjelaskan teori-teori, Anda menjelaskan apa kaitan teori tersebut dengan penelitian Anda.
Hal ini tentunya membantu pembaca dalam memahami bahwa teori yang Anda paparkan memang penting untuk memahami penelitian Anda nantinya.}

\begin{figure}
	\centering
	\includegraphics[width=\textwidth]{assets/pics/research_concept_map.png}
	\caption{Keterkaitan konsep hasil studi literatur terhadap penelitian}
	\label{fig:research_concept_map}
\end{figure}

\noindent\todo{
	Jelaskan \pic~\ref{fig:research_concept_map} di sini.
	Setiap gambar pada tugas akhir butuh penjelasan.
	Gambar hadir untuk mempermudah membaca memahami konteks, tetapi tidak bisa berdiri sendiri tanpa penjelasan.
	Terkait gambar, Anda juga bisa mengatur skalanya.
	Gambar kali ini lebarnya 0,8x dari lebar teks halaman.
}
