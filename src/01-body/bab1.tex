%-----------------------------------------------------------------------------%
\chapter{\babSatu}
\label{bab:1}
%-----------------------------------------------------------------------------%


%-----------------------------------------------------------------------------%
\section{Latar Belakang}
\label{sec:latarBelakang}
%-----------------------------------------------------------------------------%
Seiring berkembangnya teknologi informasi, pertukaran informasi antar komputer menjadi sebuah permasalahan yang semakin krusial. 
Setiap orang yang berada di dalam dunia teknologi informasi pasti tidak akan terlepas dengan jaringan komputer, dan pemahaman 
terhadap jaringan komputer sudah menjadi suatu hal yang diekspektasikan dari siapapun yang bekerja di dunia teknologi informasi. 
Dalam hal ini, Fakultas Ilmu Komputer telah menyediakan mata kuliah Jaringan Komputer Lanjut untuk mengembangkan pemahaman mahasiswa 
Fakultas Ilmu Komputer terhadap jaringan komputer dengan lebih dalam. Tentunya, mahasiswa juga perlu memiliki pemahaman yang mendasar 
terhadap jaringan komputer untuk bisa melaksanakan pembelajaran dalam mata kuliah Jaringan Komputer Lanjut secara efektif. \par

Dari kondisi tersebut, memastikan bahwa setiap mahasiswa memiliki kompetensi yang mencukupi untuk mendalami permasalahan jaringan komputer 
merupakan sebuah tantangan yang perlu ditangani. Untuk mempermudah proses tersebut, dosen Jaringan Komputer Lanjut dapat memanfaatkan sebuah 
sistem aplikasi web dimana mahasiswa dapat mengerjakan lab yang menuntut mereka untuk menunjukkan dan mempraktikkan pemahaman mereka 
terhadap jaringan komputer. Lab dalam sistem ini menghubungkan mahasiswa dengan sebuah mesin virtual yang terhubung kedalam sebuah jaringan 
virtual, dan setiap perbuatan mahasiswa dalam mesin virtual akan dicek dan dinilai melalui pengecekan log sistem operasi yang terpasang pada 
mesin virtual tersebut. 
Tentunya, diperlukan juga sebuah sistem instance dan partisi resource yang memungkinkan sistem untuk digunakan dengan jumlah pengguna yang banyak.
Setiap mahasiswa yang ingin mengerjakan sebuah lab akan diberikan sebuah instance yang dibuat oleh sistem berdasarkan suatu template yang telah didefinisikan sebelumnya, 
sehingga setiap lingkungan lab tidak memiliki keterhubungan satu sama lain.
\par

Untuk mengimplementasi sistem tersebut, aplikasi web yang akan dikembangkan perlu menerapkan sebuah teknik yang bernama Network 
Virtualization. Network Virtualization merupakan sebuah konsep yang telah berkembang pesat dalam waktu 20 tahun terakhir. Network 
Virtualization itu sendiri merupakan sebuah konsep atau proses yang memungkinkan absktraksi dari fungsi-fungsi perangkat keras dalam 
sebuah jaringan komputer kedalam perangkat lunak, sehingga sebuah Virtual Network yang berfungsi layaknya sebuah jaringan komputer 
tradisional yang menghubungkan lebih dari satu perangkat lunak dapat disimulasikan dan dijalankan dalam satu perangkat keras. 
\footnote{Carapinha, Jorge, and Javier Jiménez. “Network Virtualization: A View from the Bottom.” Proceedings of the 1st ACM workshop on Virtualized infrastructure systems and architectures (2009). https://doi.org/10.1145/1592648.1592660. }
\footnote{Tutschku, Kurt, Thomas Zinner, Akihiro Nakao, and Phuoc Tran-Gia. "Network virtualization: Implementation steps towards the future internet." Electronic Communications of the EASST 17 (2009).}
\footnote{Yi, Bo, Xingwei Wang, Keqin Li, and Min Huang. "A comprehensive survey of network function virtualization." Computer Networks 133 (2018): 212-262. https://doi.org/10.1016/j.comnet.2018.01.021}
\par

Virtual Network itu sendiri secara umum merupakan sebuah teknologi atau konsep yang menghubungkan mesin dan/atau perangkat virtual 
menggunakan perangkat lunak. 
\footnote{What is virtual networking? | vmware glossary, accessed January 19, 2024, https://www-stage-akorig.vmware.com/topics/glossary/content/virtual-networking.html.} 
Virtual Network memungkinkan mesin virtual untuk berkomunikasi dengan jaringan, host machine, dan mesin virtual lainnya tanpa adanya perangkat keras sebagai penghubung mesin-mesin tersebut layaknya dalam sebuah jaringan komputer tradisional. 
\footnote{Thomas Olzak, Jason Boomer, Robert M. Keefer, James Sabovik. "Managing Hyper-V" Microsoft Virtualization (2010): 39-60. https://doi.org/10.1016/B978-1-59749-431-1.00004-7} 
Terdapat beberapa klasifikasi dan tipe dari Virtual Network. Untuk sistem yang akan dikembangkan, terdapat satu kelas dari Virtual Network yang relevan, yaitu Virtual Private Network (VPN).
VPN itu sendiri didefinisikan secara umum sebagai sebuah metode untuk menghubungkan dua jaringan yang terpisah melalui sebuah koneksi privat melalui koneksi atau jaringan publik seperti internet.
\footnote{Ferguson, Paul, and Geoff Huston. "What is a VPN?." (1998): 01-22.}
\par

Dari konsep dan teknologi yang telah disebutkan sebelumnya, kita dapat menentukan konsep dari solusi yang dapat diimplementasi untuk sistem yang diinginkan.
Sistem yang nantinya dikembangkan akan memanfaatkan konsep Network Virtualization untuk mengakomodasi kebutuhan dari pembuatan lingkungan lab virtual.
Untuk mengimplementasikan sebuah sistem instance lab yang terisolasi dan tidak memiliki keterhubungan satu sama lain, 
dapat digunakan teknologi VPN dimana setiap instance akan berada pada Virtual Network yang berbeda.
Pengguna kemudian dapat mengakses lab dengan melakukan koneksi terhadap virtual machine pada lingkungan lab melalui koneksi VPN.
\par


%-----------------------------------------------------------------------------%
\section{Permasalahan}
\label{sec:masalah}
%-----------------------------------------------------------------------------%
\noindent\todo{Sebutkan permasalahan penelitian Anda dari latar belakang tersebut.}

%-----------------------------------------------------------------------------%
\subsection{Definisi Permasalahan}
\label{sec:definisiMasalah}
%-----------------------------------------------------------------------------%
Berikut ini adalah rumusan permasalahan dari penelitian yang dilakukan:
\begin{itemize}
	\item Bagaimana cara membuat pertanyaan penelitian?
\end{itemize}
\noindent\todo{Tuliskan permasalahan yang ingin diselesaikan. Bisa juga berbentuk pertanyaan}

%-----------------------------------------------------------------------------%
\subsection{Batasan Permasalahan}
\label{sec:batasanMasalah}
%-----------------------------------------------------------------------------%
Berikut ini adalah asumsi yang digunakan sebagai batasan penelitian ini:
\begin{itemize}
	\item Salah satu batasannya adalah, ini hanya \f{template}.
\end{itemize}

\noindent\todo{Umumnya ada asumsi atau batasan yang digunakan untuk menjawab pertanyaan-pertanyaan penelitian diatas.}


%-----------------------------------------------------------------------------%
\section{Tujuan Penelitian}
\label{sec:tujuan}
%-----------------------------------------------------------------------------%
Berikut ini adalah tujuan penelitian yang dilakukan:
\begin{itemize}
	\item Untuk memberikan \f{template} yang dapat mempermudah skripsi orang lain.
\end{itemize}

\noindent\todo{Tuliskan tujuan penelitian Anda di bagian ini.}


%-----------------------------------------------------------------------------%
\section{Posisi Penelitian}
\label{sec:posisiPenelitian}
%-----------------------------------------------------------------------------%
\todo{
	Sebutkan posisi penelitian Anda. Ada baiknya jika Anda menggunakan gambar atau diagram.
	Template ini telah menyediakan contoh cara memasukkan gambar.
	}

\begin{figure}
	\centering
	\includegraphics[width=0.4\textwidth]{assets/pics/makara.png}
	\caption{Penjelasan singkat terkait gambar.}
	\label{fig:research_position}
\end{figure}

\noindent\todo{Jelaskan \pic~\ref{fig:research_position} di sini.}


%-----------------------------------------------------------------------------%
\section{Langkah Penelitian}
\label{sec:langkahPenelitian}
%-----------------------------------------------------------------------------%
Berikut ini adalah langkah penelitian yang telah dilakukan:
\begin{enumerate}
	\item Tinjauan literatur \\
	Pada tahap ini, dipelajari teori-teori yang terkait dengan penelitian ini untuk mendapatkan konsep dasar yang dibutuhkan dalam mencapai tujuan penelitian.
	\item Analisis implementasi dan kesimpulan \\
	Pada tahap ini, digunakan studi kasus untuk analisis terkait kegunaan \f{template}.
	Setelah melakukan analisis tersebut, ditarik kesimpulan keseluruhan dari penelitian ini.
\end{enumerate}


%-----------------------------------------------------------------------------%
\section{Sistematika Penulisan}
\label{sec:sistematikaPenulisan}
%-----------------------------------------------------------------------------%
Sistematika penulisan laporan adalah sebagai berikut:
\begin{itemize}
	\item Bab 1 \babSatu \\
	    Bab ini mencakup latar belakang, cakupan penelitian, dan pendefinisian masalah.
	\item Bab 2 \babDua \\
	    Bab ini mencakup pemaparan terminologi dan teori yang terkait dengan penelitian berdasarkan hasil tinjauan pustaka yang telah digunakan, sekaligus memperlihatkan kaitan teori dengan penelitian.
	\item Bab 3 \babTiga \\
	    Apa itu Bab 3?
	\item Bab 4 \babEmpat \\
		Apa itu Bab 4?
	\item Bab 5 \babLima \\
	    Apa itu Bab 5?
	\item Bab 6 \kesimpulan \\
	    Bab ini mencakup kesimpulan akhir penelitian dan saran untuk pengembangan berikutnya.
\end{itemize}

\noindent\todo{Anda bisa mengubah atau menambahkan penjelasan singkat mengenai isi masing-masing bab. Setiap tugas akhir pasti ada yang berbeda pada bagian ini.}
