%-----------------------------------------------------------------------------%
\chapter{\babSatu}
\label{bab:1}
%-----------------------------------------------------------------------------%


%-----------------------------------------------------------------------------%
\section{Latar Belakang}
\label{sec:latarBelakang}
%-----------------------------------------------------------------------------%
Peran dari internet di abad ke-21 sudah tidak bisa dipisahkan lagi dari kehidupan sehari-hari manusia. Perkembangan teknologi yang terjadi
dalam bidang teknologi informasi memungkinkan terjadinya peningkatan pesat terhadap aksesibilitas dari perangkat-perangkat yang memungkinkan kita untuk
dapat terhubung satu sama lain meskipun berada dalam jarak fisik yang jauh melalui jaringan internet. Menurut survei yang dilakukan oleh APJII (2024),
terdapat sebanyak 221 juta pengguna internet di Indonesia, yang berarti penetrasi pengguna internet di Indonesia telah mencapai angka 79,5\% dari total
jumlah penduduk Indonesia. 
\par

Relevansi dari internet dalam kehidupan sehari-hari penduduk Indonesia menyebabkan jaringan komputer menjadi salah satu aspek yang penting dalam perkembangan
teknologi informasi di Indonesia kedepannya. Untuk itu, Indonesia perlu menyiapkan sumber daya manusia yang memiliki kompetensi yang memadai dalam bidang
jaringan komputer. Fakultas Ilmu Komputer Universitas Indonesia telah menyediakan sarana untuk pemberdayaan kompetensi tersebut melalui pengadaan
mata kuliah Jaringan Komputer dan Jaringan Komputer Lanjut yang ditawarkan kepada mahasiswa Universitas Indonesia. Mata kuliah Jaringan Komputer Lanjut
hadir sebagai lanjutan dari mata kuliah Jaringan Komputer, dimana mahasiswa Jaringan Komputer Lanjut diharapkan telah memahami dan memiliki kompetensi
yang memadai dalam materi-materi yang telah disediakan dalam mata kuliah Jaringan Komputer.
\par

Memastikan bahwa mahasiswa Jaringan Komputer Lanjut memiliki kompetensi yang memadai dalam materi mata kuliah Jaringan Komputer dapat dilakukan dengan berbagai macam cara.
Dalam waktu penulisan, mata kuliah Jaringan Komputer Lanjut biasanya mengalokasikan waktu satu minggu sebagai waktu untuk meninjau ulang materi mata kuliah
Jaringan Komputer, dimana mahasiswa akan diberikan materi singkat untuk mengingat kembali mata kuliah Jaringan Komputer serta akan diberikan sebuah evaluasi
dalam bentuk kuis untuk mengetes sebarapa jauh mahasiswa memahami materi Jaringan Komputer. Evaluasi kuis ini dilaksanakan dalam bentuk tes tertulis, yang nantinya
akan dinilai oleh dosen secara manual. 
\par

Meskipun proses tersebut merupakan sebuah solusi yang memadai untuk meninjau kompetensi mahasiswa, tentunya proses tersebut juga membutuhkan waktu yang cukup lama,
yang akhirnya berdampak dalam penundaan pemberian materi mata kuliah Jaringan Komputer Lanjut itu sendiri. Penulis mengusulkan sebuah solusi yang dapat mempercepat
dan mempermudah proses peninjauan kompetensi mahasiswa Jaringan Komputer Lanjut melalui pengembangan sistem aplikasi web yang memberikan layanan lab
untuk mahasiswa, yang nantinya akan dimanfaatkan sebagai saran penilaian otomatis dan menggantikan evaluasi tes tertulis yang sebelumnya digunakan pada mata kuliah Jaringan Komputer Lanjut.
Sistem ini akan dibangun dengan memerhatikan aspek-aspek \textit{lab based learning} pada mahasiswa untuk memastikan bahwa hasil dari evaluasi yang dilakukan dengan penggunaan
lab merupakan hasil yang representatif terhadap kompetensi mahasiswa pada bidang tersebut.
\par

Dalam sistem aplikasi web ini, semua mahasiswa Jaringan Komputer Lanjut akan diberikan sebuah akun yang nantinya akan digunakan untuk mengakses aplikasi web.
Aplikasi web akan menyediakan beberapa lab yang secara keseluruhan akan mencakup seluruh materi Jaringan Komputer. Dalam sistem ini, lab yang diberikan berbentuk
mesin virtual yang dapat diakses oleh mahasiswa melalui \textit{web browser}. Mesin virtual ini telah dikonfigurasikan berdasarkan kebutuhan masing-masing topik lab
oleh dosen melalui sebuah halaman perantara pada sistem aplikasi web, dan mesin virtual ini berada di dalam sebuah jaringan virtual sehingga mesin virtual
seolah-olah berada dalam sebuah jaringan fisik dengan mesin-mesin lainnya. Mahasiswa akan diberikan sebuah halaman pada aplikasi web dimana mahasiswa dapat melihat
dan memilih seluruh lab yang tersedia. Ketika mahasiswa memutuskan untuk melaksanakan suatu lab tertentu, maka sistem akan memberikan informasi terkait tujuan dan
tugas yang perlu dilakukan oleh mahasiswa pada lab tersebut, kemudian sistem akan menyiapkan sebuah mesin virtual baru yang nantinya akan ditampilkan kepada
mahasiswa dalam \textit{web browser} mahasiswa. Mahasiswa memiliki keleluasaan untuk berinteraksi dengan mesin virtual layaknya mahasiswa berinteraksi dengan sebuah mesin fisik. 
\par

Setiap interaksi atau perintah yang dilakukan oleh mahasiswa pada mesin virtual akan dicatat oleh sistem untuk dievaluasi secara otomatis setelah lab selesai dilaksanakan.
Dosen dapat mengkonfigurasikan kondisi-kondisi yang digunakan oleh sistem untuk penilaian otomatis. Kondisi-kondisi yang dapat dikonfigurasi meliputi hal-hal seperti mengecek
apakah mahasiswa memberikan perintah tertentu dalam sesi lab tersebut, mengecek apakah koneksi antara dua \textit{node} pada jaringan virtual berhasil dibuat, dan lain-lain. 
Sistem ini memastikan bahwa setiap mahasiswa dapat mengimplementasikan konsep-konsep yang dipelajari pada mata kuliah Jaringan Komputer pada sebuah mesin yang terhubung pada sebuah jaringan,
dan sistem ini juga memastikan bahwa implementasi yang dilakukan oleh mahasiswa memang sesuai dengan implementasi yang diharapkan oleh dosen dengan memanfaatkan pengecekan kondisi-kondisi
yang telah ditetapkan sebelumnya. Dalam melaksanakan suatu lab, mahasiswa akan diberikan waktu pengerjaan yang dapat dikonfigurasi oleh dosen. Waktu pengerjaan akan dihitung ketika mahasiswa
memulai lab, dan lab akan dianggap selesai serta akan dievaluasi setelah mahasiswa menekan tombol khusus untuk menghentikan lab atau setelah waktu pengerjaan telah selesai.
\par




% Seiring berkembangnya teknologi informasi, pertukaran informasi antar komputer menjadi sebuah permasalahan yang semakin krusial. 
% Setiap orang yang berada di dalam dunia teknologi informasi pasti tidak akan terlepas dengan jaringan komputer, dan pemahaman 
% terhadap jaringan komputer sudah menjadi suatu hal yang diekspektasikan dari siapapun yang bekerja di dunia teknologi informasi. 
% Dalam hal ini, Fakultas Ilmu Komputer telah menyediakan mata kuliah Jaringan Komputer Lanjut untuk mengembangkan pemahaman mahasiswa 
% Fakultas Ilmu Komputer terhadap jaringan komputer dengan lebih dalam. Tentunya, mahasiswa juga perlu memiliki pemahaman yang mendasar 
% terhadap jaringan komputer untuk bisa melaksanakan pembelajaran dalam mata kuliah Jaringan Komputer Lanjut secara efektif. \par

% Dari kondisi tersebut, memastikan bahwa setiap mahasiswa memiliki kompetensi yang mencukupi untuk mendalami permasalahan jaringan komputer 
% merupakan sebuah tantangan yang perlu ditangani. Pada saat proses penulisan, solusi dari tantangan tersebut merupakan pengadaan materi
% \textit{review} oleh dosen Jaringan Komputer Lanjut yang dilaksanakan melalui pemberian materi ulang serta penilaian dalam bentuk kuis dan tes tertulis.
% Proses ini membutuhkan verifikasi manual yang memakan waktu satu minggu yang seharusnya dapat dimanfaatkan untuk mendalami materi-materi yang perlu dibahas
% dalam mata kuliah Jaringan Komputer Lanjut. Untuk mempermudah dan mempercepat proses yang memakan waktu ini, dapat dikembangkan sebuah 
% sistem aplikasi web dimana proses evaluasi kompetensi dilakukan melalui pengerjaan \textit{lab online} yang memiliki sistem penilaian otomatis seperti \textit{grader}.
% Lab dalam sistem ini menghubungkan mahasiswa dengan sebuah mesin virtual yang terhubung kedalam sebuah jaringan 
% virtual atau \textit{virtual network}, dan setiap perbuatan mahasiswa dalam mesin virtual akan dicatat dan dinilai melalui pengecekan log dari perintah yang telah diberikan
% oleh mahasiswa pada lab tersebut. 
% Tentunya, diperlukan sebuah sistem \textit{instance} dan partisi \textit{resource} yang memungkinkan sistem tersebut untuk digunakan dengan jumlah pengguna yang banyak.
% Setiap mahasiswa yang ingin mengerjakan sebuah lab akan diberikan sebuah \textit{instance} mesin virtual unik yang disediakan oleh sistem berdasarkan suatu \textit{template} yang telah didefinisikan sebelumnya, 
% sehingga setiap lingkungan lab tidak memiliki keterhubungan satu sama lain.
% \par

% Terdapat berbagai macam solusi sistem yang dapat dipilih dan dikembangkan untuk memenuhi kebutuhan-kebutuhan tersebut. Salah satu konsep yang dapat diterapkan
% adalah \textit{virtual network emulation}. 

% Terdapat banyak solusi yang dapat diterapkan agar sistem yang diinginkan dapat terealisasi. Konsep yang dipilih sebagai solusi dari permasalahan ini adalah konsep Network
% Virtualization. Untuk memahami apa itu \textit{Network Virtualization}, kita perlu memahami konsep \textit{virtualization}.
% \textit{Virtualization} secara prinsip meliputi penggunaan sebuah \textit{software layer} yang mengenkapsulasi sebuah sistem operasi, sehingga
% \textit{software layer} tersebut dapat memberikan input, output, dan \textit{behavior} yang identik dengan apa yang diharapkan dari sebuah perangkat fisik.
% \footnote{Michael Pearce, Sherali Zeadally, and Ray Hunt, “Virtualization,” ACM Computing Surveys 45, no. 2 (February 2013): 1–39, https://doi.org/10.1145/2431211.2431216.}
% \textit{Virtualization} memberikan sebuah abstraksi antara pengguna dan sumber daya fisik, sehingga pengguna seolah-olah dapat berinteraksi secara langsung dengan
% sumber daya fisik tersebut tanpa adanya hubungan atau koneksi langsung antara pengguna dan sumber daya fisik.
% \footnote{Carapinha, Jorge, and Javier Jiménez. “Network Virtualization: A View from the Bottom.” Proceedings of the 1st ACM workshop on Virtualized infrastructure systems and architectures (2009). https://doi.org/10.1145/1592648.1592660. }
% Berangkat dari pemahaman konsep \textit{virtualization} tersebut, Network Virtualization merupakan sebuah teknologi yang memungkinkan pengoperasian dari 
% banyak \textit{logical network} dalam satu perangkat fisik.
% \footnote{Tutschku, Kurt, Thomas Zinner, Akihiro Nakao, and Phuoc Tran-Gia. "Network virtualization: Implementation steps towards the future internet." Electronic Communications of the EASST 17 (2009).}
% Dengan memungkinkan adanya banyak arsitektur jaringan komputer yang bersifat heterogen dalam satu perangkat fisik, Network Virtualization memberikan flexibilitas, mempromosikan keragaman, menjamin keamanan, 
% dan meningkatkan pengelolaan.
% \footnote{N.M.M.K. Chowdhury and R. Boutaba, “Network Virtualization: State of the Art and Research Challenges,” IEEE Communications Magazine 47, no. 7 (July 2009): 20–26, https://doi.org/10.1109/mcom.2009.5183468.}
% \par

% Implementasi dari konsep atau teknologi Network Virtualization dilakukan dengan memanfaatkan Mininet. Mininet adalah sebuah \textit{network emulator} yang 
% dapat mengemulasikan sebuah jaringan virtual yang terdiri dari \textit{virtual hosts, switches, controllers,} dan \textit{links}. 
% \footnote{Mininet Project Contributors, “Mininet Overview,” Mininet, accessed February 9, 2024, https://mininet.org/overview/.}
% Mininet dipilih sebagai solusi dari \textit{emulator} jaringan virtual karena Mininet menyediakan sebuah API untuk Python yang dapat digunakan
% sebagai dasar dari \textit{backend} sistem web yang akan dikembangkan. Dengan Mininet, pembuatan sebuah jaringan virtual dapat dilaksanakan
% melalui \textit{command} yang telah disediakan oleh Mininet, dan interaksi dengan \textit{node} pada jaringan virtual tersebut dapat dilakukan
% dengan memanfaatkan API untuk Python. 

% Virtual Network itu sendiri secara umum merupakan sebuah teknologi atau konsep yang menghubungkan mesin dan/atau perangkat virtual 
% menggunakan perangkat lunak. 
% \footnote{What is virtual networking?, accessed February 9, 2024, https://www.vmware.com/topics/glossary/content/virtual-networking.html.} 
% Virtual Network memungkinkan mesin virtual untuk berkomunikasi dengan jaringan, \textit{host machine}, dan mesin virtual 
% lainnya tanpa adanya perangkat keras sebagai penghubung mesin-mesin tersebut layaknya dalam sebuah jaringan komputer tradisional. 
% \footnote{Thomas Olzak, Jason Boomer, Robert M. Keefer, James Sabovik. "Managing Hyper-V" Microsoft Virtualization (2010): 39-60. https://doi.org/10.1016/B978-1-59749-431-1.00004-7} 
% Virtual Network merupakan dasar dari sistem yang akan dikembangkan nantinya serta merupakan tujuan akhir dari pengimplementasian teknologi \textit{Network Virtualization},
% dimana setiap lab akan berada dalam sebuah lingkungan berupa Virtual Network, sehingga setiap \textit{instance} dari lab bersifat unik dan terisolasi satu sama lain.
% \par

%-----------------------------------------------------------------------------%
\section{Permasalahan}
\label{sec:masalah}
%-----------------------------------------------------------------------------%
Dalam implementasi sebuah sistem yang menyediakan fitur \textit{instance} unik untuk setiap lingkungan lab, 
Untuk mengembangkan sebuah sistem yang diinginkan, maka perlu dikembangkan fitur-fitur seperti logging untuk interaksi antara user dengan virtual machine,
sistem instance yang bersifat terisolasi satu sama lain, sistem alokasi resource untuk instance yang telah dibuat oleh sistem, sistem otentikasi dan otorisasi untuk membedakan pengguna,
sistem session untuk memastikan bahwa setiap instance tidak memakan resource secara sia-sia setelah lab selesai digunakan, dan lain-lain. 


Dalam mengembangkan sebuah sistem yang mengimplementasikan sistem instance yang bisa dibuat dalam jumlah yang banyak,
salah satu permasalahan terbesar yang akan muncul adalah terkait resource allocation.
Dalam sebuah lab yang paling sederhana, dibutuhkan dua buah virtual machine yang terhubung dalam sebuah virtual network.
Tergantung terhadap sistem operasi yang digunakan untuk virtual machine, diperlukan storage yang cukup besar hanya untuk instalasi dari sistem operasi itu sendiri.
Selain itu, tentunya resource lainnya seperti pemanfaatan CPU, GPU, dan memori RAM juga perlu dialokasikan kedalam setiap virtual machine yang telah dibuat dalam suatu instance.
Kita dapat dengan mudah membayangkan kasus dimana tidak ada resource yang mencukupi untuk memenuhi kebutuhan dari instance lab, bahkan dengan sebuah komputer yang tergolong modern dan canggih.
\par
%-----------------------------------------------------------------------------%
\subsection{Definisi Permasalahan}
\label{sec:definisiMasalah}
%-----------------------------------------------------------------------------%
Berdasarkan penjelasan pada bagian permasalahan, maka dapat didefinisikan rumusan permasalahan dari proyek ini sebagai berikut:
\begin{itemize}
	\item Bagaimana cara merancang sebuah virtual network yang dapat digunakan sebagai dasar dari sistem instance untuk lab?
	\item Bagaimana cara mengimplementasikan sebuah sistem logging dari setiap perintah yang dilakukan pada virtual machine oleh pengguna lab?
	\item Bagaimana cara mengoptimasi sistem dalam pembagian, penggunaan, dan alokasi resource yang terbatas?
	\item Apa solusi dari keterbatasan resource untuk pembuatan sistem?
\end{itemize}
-----------------------------------------------------------------------------%
\subsection{Batasan Permasalahan}
\label{sec:batasanMasalah}
%-----------------------------------------------------------------------------%
Berikut ini adalah asumsi yang digunakan sebagai batasan penelitian ini:
\begin{itemize}
	\item Salah satu batasannya adalah, ini hanya \f{template}.
\end{itemize}

\noindent\todo{Umumnya ada asumsi atau batasan yang digunakan untuk menjawab pertanyaan-pertanyaan penelitian diatas.}


%-----------------------------------------------------------------------------%
\section{Tujuan Penelitian}
\label{sec:tujuan}
%-----------------------------------------------------------------------------%
Berikut ini adalah tujuan penelitian yang dilakukan:
\begin{itemize}
	\item Mengembangkan sistem aplikasi web yang menyediakan layanan dalam bentuk lab virtual.
	\item Memudahkan proses evaluasi kompetensi mahasiswa Jaringan Komputer Lanjut.
\end{itemize}


%-----------------------------------------------------------------------------%
\section{Posisi Penelitian}
\label{sec:posisiPenelitian}
%-----------------------------------------------------------------------------%
\todo{
	Sebutkan posisi penelitian Anda. Ada baiknya jika Anda menggunakan gambar atau diagram.
	Template ini telah menyediakan contoh cara memasukkan gambar.
	}

\begin{figure}
	\centering
	\includegraphics[width=0.4\textwidth]{assets/pics/makara.png}
	\caption{Penjelasan singkat terkait gambar.}
	\label{fig:research_position}
\end{figure}

\noindent\todo{Jelaskan \pic~\ref{fig:research_position} di sini.}


% %-----------------------------------------------------------------------------%
\section{Langkah Penelitian (Metode Pemecahan Masalah)}
\label{sec:langkahPenelitian}
%-----------------------------------------------------------------------------%
Untuk memecahkan permasalahan yang telah dirumuskan, maka metode-metode berikut ini akan diterapkan dalam penelitian:
\begin{enumerate}
	\item Pendalaman materi terhadap Virtualization, Network Virtualization, Virtual Network, Virtual Machine, dan Operating System Logging. \\
	Sebelum pengembangan aplikasi dapat dilaksanakan, perlu adanya pemahaman yang mendalam terhadap konsep-konsep yang akan diterapkan sebagai sistem solusi dari permasalahan.
	\item Penentuan infrastruktur solusi \\
	Setelah pendalaman materi telah dilaksanakan, perlu dirumuskan sebuah solusi konkrit yang mencakup infrastruktur dan stack yang akan digunakan untuk pengembangan sistem.
	\item Pengembangan sistem \\
	Apabila infrastruktur dari sistem solusi telah ditentukan, maka pengembangan aplikasi berdasarkan perancangan yang telah dibuat dapat dilaksanakan.
\end{enumerate}


%-----------------------------------------------------------------------------%
\section{Sistematika Penulisan}
\label{sec:sistematikaPenulisan}
%-----------------------------------------------------------------------------%
Sistematika penulisan laporan adalah sebagai berikut:
\begin{itemize}
	\item Bab 1 \babSatu \\
	    Bab ini mencakup latar belakang, cakupan penelitian, dan pendefinisian masalah.
	\item Bab 2 \babDua \\
	    Bab ini mencakup pemaparan terminologi dan teori yang terkait dengan penelitian berdasarkan hasil tinjauan pustaka yang telah digunakan, sekaligus memperlihatkan kaitan teori dengan penelitian.
	\item Bab 3 \babTiga \\
	    Apa itu Bab 3?
	\item Bab 4 \babEmpat \\
		Apa itu Bab 4?
	\item Bab 5 \babLima \\
	    Apa itu Bab 5?
	\item Bab 6 \kesimpulan \\
	    Bab ini mencakup kesimpulan akhir penelitian dan saran untuk pengembangan berikutnya.
\end{itemize}

\noindent\todo{Anda bisa mengubah atau menambahkan penjelasan singkat mengenai isi masing-masing bab. Setiap tugas akhir pasti ada yang berbeda pada bagian ini.}
